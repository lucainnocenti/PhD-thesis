%! TEX program = xelatex

\chapter{Quantum Gate Learning}
\label{Section:GateLearning}

Synthesising target quantum operations is pivotal is a number of contexts in quantum information science, for example in quantum simulation~\cite{feynman1982simulating,lloyd1996universal}, quantum computation~\cite{feynman1982simulating,deutsch1985quantum,gottesman1998theory,nielsen2002quantum,ladd2010quantum}.
In particular, unitary gates play a key role in the majority of schemes for quantum computation and quantum algorithms, and more generally are a fundamental component of most quantum information protocols.

Implementing a given target unitary gate can however often be a daunting task. Different techiniques can be used to achieve this, depending on the particular experimental context and constraints.
For example, in a photonics context, one often has access to a toolbox of elementary components, such as beamsplitters and phaseshifters, which are used to build up more complex operations.
In this context, to built a complex operation $U$, one is tasked with finding a suitable combination of such elementary (generally unitary) operations $G_i$ such that $U=\prod_i G_i$. Given a fixed set of \textit{elementary gates} $\{G_i\}_i$, finding a suitable combination of these gates that results in the target operation $U$ is highly nontrivial, and is a task often referred to as the \textit{gate compilation} or \textit{gate synthesis} problem~\cite{mottonen2004quantum,nielsen2006quantum,loke2014optqc,loke2016optqc,nakajima2009synthesis,maslov2017basic,swaddle2017generating}. \highlight{(add more citations)}
\highlight{are there other experimental contexts in which it is common to have such \textit{black boxes} that are composed together?}
Solving gate compilation problems is especially important in the light of the recent significant experimental advances in the construction of experimental quantum devices, especially superconducting~\cite{devoret2013superconducting} and ion trap~\cite{blatt2008entangled,debnath2016demonstration} architectures.

A completely different type of problem is the one faced when, instead of a discrete set of gates, one is able to tune continuous parameters of an underlying dynamics.
In other words, it can be the case that the experimenter has access to some parameters $\bs\lambda$ of the system \textit{Hamiltonian} $H_{\bs\lambda}$, and wants to find the values of $\bs\lambda$ to achieve some target dynamics.
One might for example be interested in driving a fixed input to a fixed target output, or in obtaining a full effective evolution via tuning the dynamics.
In this kind of scenario, it is also common to allow for \textit{time-dependent} dynamics. This makes it much easier, in general, to achieve different types of evolutions, but at the same times makes the resulting optimisation problem that much harder, as one then has to look for an optimal \textit{function} $t\mapsto\bs\lambda(t)$, rather than just an optimal value $\bs\lambda\in\mathbb R^n$.
This types of problems are usually referred to as \textit{quantum (optimal) control} problems.
\highlight{quantum control refs}



\begin{itemize}
    \item \textbf{\textit{You can also mix these types of problems together, and/or add ancillary resources.}}
    \item The third situation is ours: we want to tune the dynamics to getter better black boxes.
    \item What's the advantages of doing it our way?
\end{itemize}

We are \cite{innocenti2018supervised}.

Open problems and solutions, literature.

\section{Standard techniques for gate engineering, previous work}
\highlight{I'm not actually sure we want this section}

\section{Time-independent dynamics for target gates}

Here, we explore a different approach to devise target unitary evolutions. More specifically, we ask the following question: is there a way to implement a given target gate \textit{without} decomposing it as a sequence of simpler gates, \textit{and} without using a time-dependent dynamics, \textit{and} without the aid of additional ancillary resources?
In other words, given a target operation $\calU$, and a set of possible candidate Hamiltonians $\calH[\bullet]\equiv \{\calH_{\bs\lambda}\}_{\bs\lambda}\subseteq\on{Herm}$\footnote{Here, $\on{Herm}\equiv\on{Herm}(N)$ is the set of Hermitian $N\times N$ matrices with complex entries, for some integer $N\ge2$. The dimension $N$ will be taken to be understood from the context.},
can we find a \textit{time-independent} Hamiltonian $\calH\in\calH[\bullet]$ such that, at some time $t$, we have $\calU = e^{-it \calH}$?
\footnote{Throughout this work, we will only consider the cases of operations and Hamiltonians acting on \textit{finite-dimensional} systems (generally sets of qubits).}
Note that if no restriction is imposed on $\calH[\bullet]$, then the problem is always trivially solvable. Indeed, writing $\calU$ in its eigendecomposition as
$\calU=\sum_k e^{i\varphi_k}\PP_k$ for some set of orthogonal projectors $\PP_j\PP_k=\delta_{jk}\PP_j$ with $\sum_k\PP_k=I_N$ and phases $\varphi_k\in\RR$, then any $\calH$ of the form
$\calH = -\sum_k \varphi_k\mathbb P_k$ satisfies $e^{-i\calH}=\cal U$.
\highlight{Do we want to prove this linear algebra stuff at some point?}
The interesting cases, both from a theoretical and practical perspective, are therefore when $\calH[\bullet]$ is constrained to have some specific structure.
Here, we will consider the case of $\calH[\bullet]$ being a set of Hamiltonians parametrised by a number of continuous parameters (in other words, it's taken to be a parametrised surface in $\on{Herm}(N)$).
We will therefore always assume that there is a continuous mapping $\RR^\ell\ni\bs\lambda\mapsto\calH(\bs\lambda)\in\on{Herm}$ such that $\calH[\bullet]=\calH(I)$ for some $I\subseteq\RR^\ell$ (which will be usually taken to equal $\RR^\ell$ for simplicity).

We can also consider the time $t$ as part of the definition of $\calH$, which amounts to a rescaling of the energy levels. We can therefore also assume $t=1$ in the following. Finally, to ease the calculations, we will consider the equivalent problem of finding $\calH$ such that $\calU=e^{i\cal H}$ (with the plus sign), as the solutions to one can be seamlessly translated into solutions to the other problem.

For concreteness, let us analyse what happens when we try to find Hamiltonian generators for a few common two- and three-qubit gates, namely the CNOT and the Toffoli gate.
\highlight{Ensure no other gates are added in the examples}
\begin{example}[label=ex:eigendecomposition_cnot]
Assume that the target operation is $\CNOT$: the two-qubit unitary gate which flips the second qubit conditionally to the value of the first one.
In matrix notation, this reads
\begin{equation}
    \on{CNOT} \equiv\begin{pmatrix}1&0&0&0 \\ 0&1&0&0 \\ 0&0&0&1\\0&0&1&0\end{pmatrix} =
    \PP_0\otimes I_2 + \PP_1\otimes X,
\end{equation}
where $\PP_k\equiv\ketbra k$ and $X$ is the Pauli $X$ gate.
The eigendecomposition of this matrix is
\begin{equation}
    \on{CNOT} =
    \ketbra{0,0} + \ketbra{0,1} + \ketbra{1,+} - \ketbra{1,-}.
    \label{eq:cnot_eigendecomposition}
\end{equation}
A more expressive way to write~\cref{eq:cnot_eigendecomposition} is by introducing the projectors $X^\pm\equiv(1\pm X)/2$, and the similarly defined $Y^\pm$ and $Z^\pm$ (here, $X,Y,Z$ denote the one-qubit Pauli matrices).
\Cref{eq:cnot_eigendecomposition} is then equivalently written as
\begin{equation}
    \on{CNOT} =
    Z^+_1 Z^+_2 + Z^+_1 Z^-_2
    + Z^-_1 X^+_2
    - Z^-_1 X^-_2
    = Z_1^+ + Z_1^- X_2^+ - Z_1^- X_2^-.
    \label{eq:cnot_eigdecomp_with_paulis}
\end{equation}
The eigenvalues are therefore $\lambda_{1,2,3}=1$ and $\lambda_4=-1$.
Can this gate be obtained via a two-qubit time-independent Hamiltonian $\calH$?
Consider for the purpose a general parametrisation of the possible two-qubit Hamiltonians, which we can write using the set of Pauli matrices on the two qubits as operatorial basis:
\begin{equation}
    \calH({\bs\lambda}) =
    \sum_{j,k=0}^4 \lambda_{j,k} \sigma^j_1\sigma^k_2.
\end{equation}
Given the decomposition of~\cref{eq:cnot_eigdecomp_with_paulis}, it is natural to use as an Ansatz for $\calH(\bs\lambda)$ an expression bearing the same eigenstructure as the CNOT. Let us therefore write
\begin{equation}
    \calH(\bs\lambda) = \lambda_1 Z_1^+ + \lambda_2 Z_1^- X_2^+ + \lambda_3 Z_1^- X_2^-.
\end{equation}
Finding $\bs\lambda\equiv(\lambda_1,\lambda_2,\lambda_3)$ such that $e^{i\calH}=\CNOT$ is then quite easy: just use $\lambda_1=\lambda_2=0$ and $\lambda_3=2\pi$. Indeed, as it is easy to check, $e^{\pi i Z_1^- X_2^-} = \CNOT$.
The natural next question is then: is this the \textbf{only} such $\calH$? It is straightforward to see that the answer is negative: for example, it is also true that $e^{\pi i n Z_1^- X_2^-}=\CNOT$ for any $n\in\ZZ$.
It is less trivial to get a complete characterisation of the solution set.
Indeed, as will be shown in~\cref{sec:solutions_matrix_equation_f(A)=B}, there is a rich set of solutions for these types of matrix equations.
\end{example}

\begin{example}[label={ex:eigendecomposition_Toffoli}]
Consider now the \textbf{Toffoli gate}, which is a three-qubit gate which flips the third qubit conditionally to the first two qubits being in the $\ket1$ state.\highlight{Ensure Toffoli was defined before, or add relevant references here.}
More explicitly, this means
\begin{equation}
    \Toff =
    \PP_0\otimes I_4 + \PP_1\otimes\CNOT
    \equiv
    \begin{pmatrix}
        I_4 & \mathbb{0}_4 \\
        \mathbb 0_4 & \CNOT
    \end{pmatrix}.
    % \begin{pmatrix}
    %     1&0&0&0&0&0&0&0 \\
    %     0&1&0&0&0&0&0&0 \\
    %     0&0&1&0&0&0&0&0 \\
    %     0&0&0&1&0&0&0&0 \\
    %     0&0&0&0&1&0&0&0 \\
    %     0&0&0&0&0&1&0&0 \\
    %     0&0&0&0&0&0&0&1 \\
    %     0&0&0&0&0&0&1&0
    % \end{pmatrix}
\end{equation}
In terms of projectors onto eigenvalues of (products of) Pauli matrices, we have
\begin{equation}
    \Toff = Z_1^+ + Z_1^- Z_2^+ + Z_1^- Z_2^- X_3^+ - Z_1^- Z_2^- X_3^-.
\end{equation}
The eigenvalues of $\Toff$ are thus readily seen to be $+1$ with multiplicity $7$, and $-1$ with multiplicity $1$.
A possible $\calH$ such that $e^{i\cal H}=\Toff$ is then $\calH=\pi Z_1^- Z_2^- X_3^-$.
Note how expanding this Hamiltonian we get
\begin{equation}
    \calH = \frac{\pi}{8}\left[
        I - (Z_1 - Z_2 - X_3)
        + (Z_1 Z_2 + Z_1 X_3 + Z_2 X_3)
        - Z_1 Z_2 X_3
    \right],
\end{equation}
which contains three-qubit interaction terms (the $Z_1 Z_2 X_3$ factor). While this is to be expected , as the Toffoli is a non-trivial three-qubit gate, these kinds of terms make practically implementing these Hamiltonians significantly harder.
Indeed, natural interactions, and in particular interactions that can be easily implemented in experiments, generally \highlight{can we promote this to \textbf{always}?} involve only one- and two-qubit interaction terms.
It is then natural to wonder about whether this is a \textbf{necessary} feature of time-independent generators of the Toffoli gates.
As will be shown in the following sections, the answer is in fact that yes, it is possible to find time-independent Hamiltonians that generate a Toffoli gate \textbf{and} only use up to two-qubit interactions.
\end{example}

As shown in~\cref{ex:eigendecomposition_cnot,ex:eigendecomposition_Toffoli}, looking for Hermitian generators of few-qubit gates is not a fairly straightforward task. There are, however, two notable issues with the type of analysis conducted so far:
\begin{enumerate}
    \item It lacks in generality: while finding specific generators essentially amounts to computing the eigendecomposition of the target $\calU$, and then the logarithm of the eigenvalues, this says nothing about whether such procedure provides the most general kind of Hamiltonian generator for the given gate. In other words, can \textit{all} generators $\calH$ be obtained this way?
    \item We have no control over the kind of generators that we obtain with this procedure. For example, in~\cref{ex:eigendecomposition_cnot} we obtained generators containing two-qubit interactions of the form $Z_1 X_2$. Does this mean that this type of interaction term is \textit{necessary} to generate a CNOT? Or is there instead some Hamiltonian that can still generate the CNOT without using such terms?
    Similar questions arise in~\cref{ex:eigendecomposition_Toffoli}.
\end{enumerate}
To address these issues, we will study the underlying mathematical problem in more depth. As it turns out, we can divide such analysis in two parts. In~\cref{sec:solutions_matrix_equation_f(A)=B}, we show how to find general solutions to inverse matrix equations of the form $f(A)=B$. Then, in~\cref{sec:constraints_on_interaction_pars}, we present a way to bring additional constraints on the interaction terms into the discussion.
\highlight{to review this part probably}

\section{Solutions of the matrix equation \texorpdfstring{$f(A)=B$}{f(A)=B}}
\label{sec:solutions_matrix_equation_f(A)=B}
Let $A,B$ be normal matrices such that $f(A)=B$ for some function $f$ (assumed to be regular on the spectrum of $A$), with $f(A)$ defined as usual on the eigenvalues of $A$.
In this section we will provide a full characterisation of the set of solutions $A\in f^{-1}(B)$ .

Explicitly, $f(A)=B$ amounts to
\begin{equation}
    \sum_j \lambda_j^B \PP_j^B = \sum_k f(\lambda_k^A) \PP_k^A,
    \label{eq:f(A)=B_eigendecomp_proof}
\end{equation}
where $\PP_k^{A} (\PP_k^{B})$ are the projectors onto the corresponding eigenspaces of $A$ and $B$, so that
$A\PP_k^A=\lambda_k \PP_k^A$, and similarly for $B$.
Because both sides of~\cref{eq:f(A)=B_eigendecomp_proof} define the eigendecomposition of some (normal) operator, by the uniqueness of the eigendecompositions it follows that $\lambda_j^B=f(\lambda_j^A)$ and $\PP_j^B=\PP_j^A$.
What is interesting about this is that it means that the relation $f(A)=B$ puts strong constraints on the eigenstructure of $A$: the eigenspaces of $A$ must be the same as those of $B$.

Consider now the problem of finding a matrix $A$ such that $f(A)=B$, for some predetermined mapping $f$.
If $f$ is injective (at least on $f^{-1}(\sigma(B))$), then, by our above argument, it follows that $A$ is uniquely determined. Indeed, we the condition would then be
\begin{equation}
    \sum_j \lambda_j^B \PP_j = \sum_j f(\lambda_j^A) \PP_j,
\end{equation}
which can only be true for $\lambda_j^A = f^{-1}(\lambda_j^B)$.
However, because we are considering here matrices and functions defined over $\CC$, most interesting cases will correspond to \textit{non-injective} functions $f$\footnote{Indeed, the only entire, injective functions $\CC\to\CC$ are the linear mappings $z\mapsto az+b$ for some $a,b\in\CC$.}. The case $f(z)=e^z$, which is the one with which we will be mostly concerned with for the gate learning problem, is one such case of non-injective function.

When $f$ is non-injective, there can be multiple $A$ such that $f(A)=B$. There are two main reasons for this, one fairly evidence, and the other one less so.
Clearly, being $f$ not injective, for any given eigenvalue $\lambda_j^B$ there can be multiple $\lambda$ such that $f(\lambda)=\lambda_j^B$. This is indeed the only source of non-uniqueness whenever the eigenspaces of $B$ are all one-dimensional (equivalently, the projetors $\PP_k^B$ in~\cref{eq:f(A)=B_eigendecomp_proof} have all unit trace).
However, whenever there are eigenspaces of dimension greater than one, there are multiple (infinite) ways to write a corresponding projector $\PP$ as sum of unit-trace projectors. Assume for example that $\Tr\PP=d$, and let $\PP_k$ be a set of $d$ unit-trace projectors such that $\PP=\sum_{k=1}^d \PP_k$. Let $U$ be an arbitrary operator that is unitary in the range of $\PP$.
Then we have
\begin{equation}
    \PP=U^\dagger \PP U =\sum_{k=1}^d U^\dagger \PP_k U
    \equiv \sum_{k=1}^d \tildePP_k,
    \label{eq:rewriting_proj_with_rotated_projs}
\end{equation}
where we defined the ``rotated projectors'' $\tildePP_k\equiv U^\dagger \PP U$.
While this rewriting is ininfluent in~\cref{eq:rewriting_proj_with_rotated_projs}, such rotations of the degenerate eigenspaces turn out to be of great relevance when looking for solutions of $f(A)=B$.
To see this, let us consider again~\cref{eq:f(A)=B_eigendecomp_proof}, and focus on a single eigenspace $\PP_j^B$ such that $\Tr\PP_j^B=d_j>1$ (assuming $B$ has one), and write an arbitrary decomposition of $\PP_j^B$ in terms of on unit-trace projectors as $\PP_j^B=\sum_{\ell=1}^{d_j}\PP^B_{j,\ell}$.
Let $\{\lambda_{j,\ell}\}_{\ell=1}^{d_j}\subseteq f^{-1}(\lambda_j^B)$ be a subset of inverses of $\lambda_j^B$. Then, any $A$ which has $\lambda_{j,\ell}$ as eigenvalues within the eigenspace $\PP_j^B$ is a suitable solution for $f^{-1}(B)$.
In other words, we have
\begin{equation}
    A \equiv \sum_j \sum_{\ell=1}^{d_j} \lambda_{j,\ell} \PP_{j,\ell}\in f^{-1}(B),
    \quad \forall \lambda_{j,\ell}\in\CC \text{ with } f(\lambda_{j,\ell})=\lambda_j^B.
    \label{eq:general_form_f-1B}
\end{equation}
Note how this means that $f^{-1}(B)$ \textit{can break the degeneracies in $B$} thanks for the non-injectivity of $f$.

Let us give now a few concrete examples of the freedom allowed by the degeneracies.

\begin{example}[label={ex:solutions_f(A)=I2}]
Consider the two-dimensional identity matrix $I_2$, and an arbitrary complex function $f$ defined on the complex unit. \textbf{What is then the set of matrices $A$ such that $f(A)=I_2$?}
Note that $I_2=P+Q$ for any pair of orthogonal projectors $P,Q$, and $\lambda P + \mu Q\in f^{-1}(I_2)$ for any $\lambda,\mu\in f^{-1}(1)$.
This can be equivalently stated saying that, for any pair of orthonormal states $\ket u$ and $\ket v$, we have
$\lambda \ketbra u + \mu \ketbra v\in f^{-1}(I_2)$.
Going further, we parametrise the set of all such vectors writing (we can define the vectors up to an overall phase, as we are only interested in the corresponding projectors):
\begin{equation}
    \begin{cases}
    \ket u =\cos\alpha\ket 0+e^{i\theta}\sin\alpha \ket 1, \\
    \ket v =-e^{-i\theta}\sin\alpha\ket 0 + \cos\alpha\ket 1.
\end{cases}
\end{equation}
This provides us with a parametrisation for the set of solutions $f^{-1}(B)$:
\begin{equation}
\begin{aligned}
    A = &\lambda\left(\cos^2(\alpha) \PP_0 + \sin^2(\alpha) \PP_1 + \sin(2\alpha)\Re[e^{i\theta}\ketbra{1}{0}]\right) \\
    +&\mu\left(
        \sin^2(\alpha) \PP_0 + \cos^2(\alpha) \PP_1 - \sin(2\alpha)\Re[e^{i\theta}\ketbra{1}{0}]
    \right),
\end{aligned}
\end{equation}
or in matrix form,
\begin{equation}
    A = \begin{pmatrix}
        \lambda \cos^2(\alpha)+\mu\sin^2(\alpha) &
        \frac12 e^{-i\theta} \sin(2\alpha) (\lambda-\mu) \\ 
        \frac12 e^{i\theta} \sin(2\alpha) (\lambda-\mu) &
        \lambda \sin^2(\alpha)+\mu\cos^2(\alpha)
    \end{pmatrix}.
    \label{eq:explicit_inverse_f(A)_matrix_form}
\end{equation}
We conclude that any such $A$ is such that $f(A)=I_2$, as long as $f(\lambda)=f(\mu)=1$, and that this form covers the set of all possible normal matrices satisfying this equation.
Whenever $\lambda=\mu$,~\cref{eq:explicit_inverse_f(A)_matrix_form} reduces to $A=\lambda I_2$, which is the trivial solution. When $\lambda\neq\mu$, however, it is less obvious that the corresponding matrix in~\cref{eq:explicit_inverse_f(A)_matrix_form} satisfies $f(A)=I_2$.
\end{example}

\begin{example}[label={ex:solutions_f(A)=I_3}]
\highlight{To decide where to put this crap.}

Consider how to split the eigenspace of the three-dimensional identity $I_3$.
This essentially amounts to the problem of parametrising a set of three orthonormal complex vectors. For the purpose, let us write them as
\begin{equation}
\begin{cases}
    \ket{u_1} = \cos(\alpha)\ket0
    + e^{i\theta}\sin(\alpha)\cos(\beta)\ket1
    + e^{i\varphi}\sin(\alpha)\sin(\beta)\ket2, \\
    \ket{u_2} = \cos(\alpha)\ket0
    + e^{i\theta}\sin(\alpha)\cos(\beta)\ket1
    + e^{i\varphi}\sin(\alpha)\sin(\beta)\ket2, \\
\end{cases}
\end{equation}
\highlight{Actually, there might not be a general nice way to parametrise such sets of vectors!}
\end{example}

\begin{example}[label={ex:solutions_e^A=I2}]
Picking up from~\cref{ex:solutions_f(A)=I2}, let us analyse the solutions to $f(A)=I_2$ when $f(z)=e^z$.
Then $f^{-1}(1)=2\pi i\ZZ$, and the set of solutions becomes
\begin{equation}
    A_{\alpha,\theta;n,m} = 2\pi i
    \begin{pmatrix}
        n \cos^2(\alpha)+m\sin^2(\alpha) &
        \frac12 e^{-i\theta} \sin(2\alpha) (n-m) \\ 
        \frac12 e^{i\theta} \sin(2\alpha) (n-m) &
        n \sin^2(\alpha)+m\cos^2(\alpha)
    \end{pmatrix},
\end{equation}
for all $n,m\in\ZZ$ and $\alpha,\theta\in\RR$.
\end{example}

\begin{example}[label={ex:cnot_generator_decomposition}]
We gave in~\cref{ex:eigendecomposition_cnot} a possible Hamiltonian generator for the CNOT gate.
Here, strong on the general characterisation of matrix inverses given in~\cref{eq:general_form_f-1B}, we work out a more general expression for the set of Hamiltonians $\calH$ such that $e^{i\calH}=\CNOT$.
Specialising~\cref{eq:general_form_f-1B} to the specific eigenstructure of $\CNOT$ (which we worked out in~\cref{ex:eigendecomposition_cnot}), we see that a generic (normal) generator for $\CNOT$ has the form
\begin{equation}
    \sum_{j=1}^3 \lambda_j \tildePP_j + \mu\PP_4,
\end{equation}
where
$\PP_4\equiv\ketbra 4\equiv\ketbra{1,-}$, the projectors 
$\{\tildePP_j\}_{j=1}^3$ define an arbitrary splitting of the degenerate eigenspace $\ker[(\CNOT-I_4)]$ into orthonormal vectors, and $\lambda_j,\mu\in\CC$ are such that $e^{i\lambda_j}=1$ and $e^{i\mu}=-1$.
These last conditions identify $\lambda_j\in2\pi\ZZ$ and $\mu\in\pi + 2\pi\ZZ$.
% \begin{equation}
%     \lambda_{j,\ell} = 2\pi n_{j,\ell},
%     \quad
%     \mu_\ell = \pi + 2\pi m_\ell,
%     \quad
%     n_{j,\ell},m_\ell\in\ZZ.
% \end{equation}
A generic generator for the CNOT can thus be written as
\begin{equation}
    \calH =
    2\pi\left[
    \sum_{j=1}^3 n_j \tildePP_j +
    \frac{1}{2} m\PP_4
    \right].
\end{equation}
It is worth stressing how the set of viable generators $\calH$ has both a discrete and a continuous degree of freedom.
\end{example}

In the rest of the thesis, we will focus on the case $f(A)\equiv e^{iA}$, which is case of relevance to find Hamiltonians generating target unitaries.

\section{Adding physical practical constraints}
\label{sec:constraints_on_interaction_pars}

In~\cref{sec:solutions_matrix_equation_f(A)=B} we described how to write the full set of solution for a matrix equation of the form $f(A)=B$.
In this section we will show how introducing further constraints on the type of interaction terms allowed in the final generator reveals a particularly daunting task.

To restrict the type of interaction terms allowed in the final generator, means to fix some set $\calP$ comprised of orthogonal Hermitian operators $\sigma_i$, and ask for a generator $\calH\in f^{-1}(\calU)$ such that $\calH$ is in the span of $\{\sigma_i\}_i$:
\begin{equation}
    \calH \in f^{-1}(\calU) \cap \on{Span}\calP.
\end{equation}
In the following, we will focus on the case  $f(A)\equiv e^{iA}$. \highlight{probably need to specify what is $f$ here}
A typical (but not necessary) choice for the $\sigma_i$ is the set of one- and two-qubit interaction terms $\calP_{\le2}$, defined as
\begin{equation}
    \calP_{\le2} \equiv \{ I, X_i,Y_i,Z_i, X_i Y_j, X_i Z_j, Y_i Z_j : \,\,i,j=1,2,3 \}.
\end{equation}
We thus have for example $X_i,Y_j\in\calP_{\le2}$ and $X_i Z_j\in\calP_{\le2}$, but $X_i Y_j Z_k\notin\calP_{\le2}$ for $i\neq j,i\neq k$ and $j\neq k$.

Given a candidate generator $\calH\in f^{-1}(\calU)$, one would therefore need to decompose $\calH$ in a basis containing the elements of $\calP$, and impose that all terms outside of this set are vanishing.
Given a target gate with eigendecomposition
$\calU=\sum_j e^{i\varphi_j}\PP_j$ with $\Tr\PP_j=d_j$ and $\sum_j d_j=2^n$ with $n$ the number of qubits, this would entail in practice finding bases for each $\on{Range}(\PP_j)$ built out of states $\{\ket*{u_{jk}}\}_{k=1}^{d_j}\subset \mathbb{CP}^{d_j}$, and integers $n_{jk}$, such that
\begin{equation}
    \sum_j\sum_{k=1}^{d_j} (\varphi_j + 2\pi n_{jk}) \PP[\ket*{u_{jk}}]
    \in \on{Span}_{\RR}\calP.
\end{equation}

\begin{example}[label={ex:cnot_physical_constraints}]
Let $\calU=\CNOT$, and consider a set of allowed interactions containing only single-qubit terms: $\calP_1\equiv\{X_i,Y_i,Z_i\}_{i\in\{1,2\}}$.
Is there an $\calH$ containing only interactions in $\calP_1$ and such that $e^{i\calH}=\CNOT$?

From a physical point of view, the answer should obviously be that no, there cannot be any such Hamiltonian, as that would mean that an entangling two-qubit gate can be obtained without having the qubits interact in any way.
Still, it is interesting to see if we can recover this result using the formalism introduced in the previous section, as a testbed to see it in action.
Consider then the decomposition given in~\cref{ex:cnot_generator_decomposition} for the CNOT:
\begin{equation}
    \calH =
    2\pi\left[
    \sum_{j=1}^3 n_j \tildePP_j +
    \frac{1}{2} m\PP_4
    \right],
\label{eq:cnotex_general_H_expr}
\end{equation}
where
$\PP_4\equiv Z_1^- X_2^-$ and $\tildePP_1,\tildePP_2,\tildePP_3$ define an arbitrary splitting of the eigenspace generated by $\ket{0,0},\ket{0,1}$ and $\ket{1,+}$.
Let us write down explicitly the decompositions of these projectors in terms of Pauli matrices:
\begin{equation}
\begin{cases}
    \tildePP_1 &= Z_1^+ Z_2^+ \,= \frac{1}{4} (I + Z_1 + Z_2 + Z_1 Z_2), \\
    \tildePP_2 &= Z_1^+ Z_2^- \,= \frac{1}{4} (I + Z_1 - Z_2 - Z_1 Z_2), \\
    \tildePP_3 &= Z_1^- X_2^+ = \frac{1}{4} (I - Z_1 + X_2 - Z_1 X_2), \\
    \PP_4 &= Z_1^- X_2^- = \frac{1}{4} (I - Z_1 - X_2 + Z_1 X_2),
\end{cases}
\end{equation}
where we made a canonical choice for the $\tildePP_i$.
Without exploiting the freedom given by the degeneracies, it is straightforward to see that it is not possible to find values for the $n_j\in\ZZ$ such annihilate the two qubit interaction terms, due to the $1/2$ factor in~\cref{eq:cnotex_general_H_expr}.
Indeed, expanding~\cref{eq:cnotex_general_H_expr} and focusing on the two-qubit interaction terms, we get
\begin{equation}
    \calH = (...) + \frac{2\pi}{4}\left[
    (n_1 - n_2) Z_1 Z_2 +
    (n_4/2 - n_3) Z_1 X_2
    \right].
\end{equation}
Remarkably, this shows how the $Z_1 X_2$ seems to be a necessary part of a generator, as there is no choice of $n_3,n_4\in\ZZ$ such that $n_4-2 n_3=0$.
Still, this does not rule out the possibility that for some other choice of $\tildePP_j$ it is possible.
The difficulty in showing this in full generality arises from the lack of a nice general expression for a arbitrary unitary rotations in more than two dimensions \highlight{maybe make sure this is actually true. What about Jarlskogwhateverthefuckitscalled decomposition?}.
In~\cref{sec:gatelearning_solution_framework} we will provide a way to answer these questions more easily.
\end{example}

\begin{example}[label={ex:toffoli_physical_constraints}]
Let $\calU=\Toff$, and consider the set of allowed interactions $\calP_{\le2}$ comprised of one- and two-qubit interaction terms (but notably \textit{not} three-qubit terms).
\textbf{\textit{Is there a generator $\calH\in f^{-1}(\Toff)\cap\calP_{\le2}$?}}

The eigenstructure of the Toffoli gate was given in~\cref{ex:eigendecomposition_Toffoli}. Similarly to the case of the CNOT, we here have a sevenfold degenerate eigenvalues $+1$ and a nondegenerate eigenvalue $-1$.
The eigenvector of the eigenvalue $-1$ is $\ket{1,1,-}$. We can therefore write a general expression for generators of $\Toff$ in the form
\begin{equation}
    \calH=2\pi \left[
        \sum_{j=1}^7 n_j \tildePP_j
        +\frac12 m\PP_8,
    \right]
\end{equation}
where $\PP_8\equiv\ketbra{1,1,-}$ and $\sum_{j=1}^7\tildePP_j+\PP_8=I_8$.
Note how the choice $n_j=0$ gives a generator proportional to $\PP_8$, which contains the three-qubit interaction term $Z_1 Z_2 X_3$, and is therefore not contained in $\calP_{\le2}$.
A canonical choice for the $\tildePP_j$ is the following:
\begin{equation}
\begin{cases}
    \tildePP_1 &= Z_1^+ Z_2^+ Z_3^+ = (...) + \frac18 Z_1 Z_2 Z_3, \\
    % = \frac{1}{8} (I + Z_1 + Z_2 + Z_3 + Z_1 Z_2 + Z_1 Z_3 + Z_2 Z_3 + Z_1 Z_2 Z_3), \\
    \tildePP_2 &= Z_1^+ Z_2^+ Z_3^- = (...) - \frac{1}{8} Z_1 Z_2 Z_3, \\
    %\,= \frac{1}{4} (I + Z_1 - Z_2 - Z_1 Z_2), \\
    \tildePP_3 &= Z_1^+ Z_2^- Z_3^+ = (...) - \frac18 Z_1 Z_2 Z_3, \\
    \tildePP_4 &= Z_1^+ Z_2^- Z_3^- = (...) + \frac18 Z_1 Z_2 Z_3, \\
    \tildePP_5 &= Z_1^- Z_2^+ Z_3^+ = (...) - \frac18 Z_1 Z_2 Z_3, \\
    \tildePP_6 &= Z_1^- Z_2^+ Z_3^- = (...) + \frac18 Z_1 Z_2 Z_3, \\
    \tildePP_7 &= Z_1^- Z_2^- X_3^+ = (...) + \frac18 Z_1 Z_2 X_3, \\
    \PP_8 &= Z_1^- Z_2^- X_3^- = (...) - \frac18 Z_1 Z_2 X_3.
\end{cases}
\end{equation}
Similarly to what we found in~\cref{ex:cnot_physical_constraints}, we have here a three-qubit interaction term, $Z_1 Z_2 X_3$, which is proportional to $(n_7 - m/2)$, and therefore cannot be removed by any choice of the integer parameters.
% \begin{equation}
%     (n_7 - m/2) Z_1 Z_2 X_3
% \end{equation}
There is, however, a marked difference between the present case and~\cref{ex:cnot_physical_constraints}: while for the CNOT we have strong physical reasons to believe that it is impossible to obtain the gate without two-qubit interaction terms, this is not the case for the Toffoli.
Indeed, while it is in principle possible to achieve a three-qubit entangling gate without using three-qubit interaction terms, although the veracity of this claim is not obvious.
Indeed, we will show in~\cref{sec:gatelearning_solution_framework} how, by appropriately rotating the degenerate eigenspace, it \textit{is} possible to achieve such a feat.
\end{example}

\section{A solution framework}
\label{sec:gatelearning_solution_framework}
As demonstrated in~\cref{sec:constraints_on_interaction_pars}, injecting physical constraints into the problem makes it significantly harder to solve.
Nevertheless, we will show in this section how a few observations can be exploited to significantly simplify the task.

Piggybacking on the results of~\cref{sec:solutions_matrix_equation_f(A)=B}, consider the general problem of looking for generators $\calH$ such that $e^{i\calH}=\calU$ for a given unitary $\calU$.
While, as previously discussed, there are in general many possible such $\calH$, there \textit{is} one ``natural'' choice that can be considered as \textit{canonical} in this context.
Indeed, while a generic $\calH$ can break the degeneracies of $\calU$, let us denote with $\calH_{\calU}$ a \textit{canonical generator}, which is one that preserves the degeneracies of $\calU$.
Note that computing such a generator is generally a straightforward task.
% Indeed, given an eigendecomposition of $\calU$ 
% As previously discussed, it is easy to find \textit{one} possible $\calH$ satisfying this requirement. Let us denote such a ``canonical'' generator with $\calH_{\calU}$.
Any other generator $\calH$ will have to be related to $\calH_{\calU}$ through the equation $e^{i\calH}=e^{i\calH_{\calU}}=\calU$.
We will show in this section that the following two conditions must be always satisfied by any such $\calH$:
\begin{itemize}
    \item Every such $\calH$ must commute with $\calH_{\calU}$.
    % : $[\calH,\calH_{\calU}]=0$.
    \item The eigenvalues of $\calH-\calH_{\calU}$ must all be integer multiples of $2\pi$.
\end{itemize}

Indeed, assume that $\calH$ and $\calH_{\calU}$ satisfy these two conditions.
% Then, as shown in~\cref{sec:solutions_matrix_equation_f(A)=B}, 
While, in general, two Hamiltonians $\calH_1,\calH_2$ with $e^{i\calH_1}=e^{i\calH_2}=\calU$ need not commute with each other, this must be the case when one of the two is a canonical generator $\calH_{\calU}$. In other words, we must always have $[\calH,\calH_{\calU}]=0$.
This is for the same reason why, for any generator $\calH$, we must have $[\calH,\calU]=0$.
Moreover, $[\calH,\calH_{\calU}]=0$ implies that
$I=e^{i\calH}e^{-i\calH_{\calU}}=e^{i(\calH-\calH_{\calU})}$.
But for this to be true, the eigenvalues of $\calH-\calH_{\calU}$ must necessarily be integer multiples of $2\pi$, which proves the first implication.

For the other direction, let us assume that, given some $\calH_{\calU}$ such that $e^{i\calH_{\calU}}=\calU$, we have $[\calH,\calH_{\calU}]=0$ and $\on{Spec}(\calH-\calH_{\calU})\subseteq2\pi\ZZ$.
Then,
\begin{equation}
    e^{i\calH} =
    e^{i\calH} e^{-i\calH_{\calU}} e^{i\calH_{\calU}} =
    e^{i(\calH-\calH_{\calU})} \calU =
    \calU.
\end{equation}
This suggests the following plan of action to solve the general problem in the presence of constraints on the Hamiltonian generators: let $\calU$ be a target gate with canonical generator $\calH_{\calU}$. Then, to find $\calH\in f^{-1}(\calU)\cap\calP$ is equivalent to find $\calH$ satisfying the following three conditions:
\begin{subequations}
\begin{gather}
    \calH\in\calP, \label{eq:3conditions_1st}\\
    [\calH,\calU]=0 \text{ (equivalently, $[\calH,\calH_{\calU}]=0$)}, \label{eq:3conditions_2nd}\\
    \on{Spectrum}(\calH-\calH_{\calU})\subseteq2\pi\ZZ \label{eq:3conditions_3rd}
\end{gather}
\label{eq:3conditions}
\end{subequations}
% \begin{enumerate}
%     \item $\calH\in\calP$,
%     \item $[\calH,\calU]=0$ (or, equivalently, $[\calH,\calH_{\calU}]=0$),
%     \item $\on{Spectrum}(\calH-\calH_{\calU})\subseteq2\pi\ZZ.$
% \end{enumerate}
To approach a given problem, we can therefore proceed as follows:
1) write a general expression for an element in the (real) span of $\calP$: $\calH=\sum_k c_k \sigma_k$ summed over all $\sigma_k\in\calP$.
2) Impose $[\calH,\calU]=0$, this immediately cuts many of the coefficients in the general expression for $\calH$.
3) Look into the remaining set of coefficients $c_k$ for a combination that satisfies the third condition.

Note how the first two conditions are easy to impose, while the third one remains difficult. To see this more concretely let us consider a few applications of this framework.

\begin{example}[label={ex:cnot_with_conditions}]
Taking up where we left off in~\cref{ex:cnot_physical_constraints}, let us see if~\cref{eq:3conditions} can give us a conclusive way to prove the impossibility of generating a CNOT with only one-qubit interactions.
A canonical generator for the CNOT is obtained by setting $n_i=0$ and $m=1$ in~\cref{eq:cnotex_general_H_expr}, which results in
$\calH_{\CNOT}=\pi Z_1^- X_2^-$.
A general form for an $\calH$ containing one-qubit interactions is
\begin{equation}
    \calH = h_0 I +
    \sum_{\alpha=1}^3 (h_1^\alpha\sigma_1^\alpha + h_2^\alpha\sigma_2^\alpha),
\end{equation}
where $\sigma_i^1\equiv X_i, \sigma_i^2\equiv Y_i, \sigma_i^3\equiv Z_i$.
Imposing $[\calH,\CNOT]=0$ removes most of the parameters, leaving us with the simplified expression
\begin{equation}
    \calH = h_0 I_4 + h_1^3 Z_1 + h_2^1 X_2.
    \label{eq:cnot_expr_after_commutativity_condition}
\end{equation}
One easy way to see this is to note that for $\calH$ to commute with $\CNOT$, the two matrices must respect each other's eigenspaces. In particular, this means that $\calH$ must preserve the nondegenerate eigenvector of $\CNOT$, which is $\ket{1,-}$.
The only one-qubit gates that do this are $I_4, Z_1$ and $X_2$, hence we arrive to~\cref{eq:cnot_expr_after_commutativity_condition}.

The question is now reduced to that of figuring out whether there are coefficients $h_0,h_1^3,h_2^1\in\RR$ such that the spectrum of $\calH-\calH_{\CNOT}$ contains nothing but integer multiples of $2\pi$.
In matrix form,~\cref{eq:cnot_expr_after_commutativity_condition} reads
\begin{equation}
    \calH = \begin{pmatrix}
        h_0 + h_1^3 & h_2^1 & 0 & 0 \\
        h_2^1 & h_0 + h_1^3 & 0 & 0 \\
        0 & 0 & h_0 - h_1^3 & h_2^1 \\
        0 & 0 & h_2^1 & h_0 - h_1^3
    \end{pmatrix},
\end{equation}
which has the four eigenvalues $h_0\pm h_1^3 \pm h_2^1$.
At the same time,
\begin{equation}
    \calH_{\CNOT} = \frac{\pi}{2}\begin{pmatrix}
        0 & 0 & 0 & 0 \\
        0 & 0 & 0 & 0 \\
        0 & 0 & 1 & -1 \\
        0 & 0 & -1 & 1
    \end{pmatrix},
\end{equation}
which has eigenvalues $\pi,0$. The eigenvalues of $\calH-\calH_{\CNOT}$ are then
\begin{equation}
\def\id{h_0}
\def\Z1{h_1^3}
\def\X2{h_2^1}
\begin{cases}
    \id + \Z1 - \X2 &= 2\pi \nu_1, \\
    \id - \Z1 + \X2 &= 2\pi \nu_2, \\
    \id + \Z1 + \X2 &= 2\pi \nu_3, \\
    \id - \Z1 - \X2 - \pi &= 2\pi \nu_4.
\end{cases}
\end{equation}
Inverting this system, we get from the first three equations
\begin{equation}
\def\id{h_0}
\def\Z1{h_1^3}
\def\X2{h_2^1}
\begin{cases}
    \id &= \pi (\nu_1 + \nu_2), \\
    \X2 &= \pi (\nu_3 - \nu_1), \\
    \Z1 &= \pi (\nu_3 - \nu_2).
\end{cases}
\end{equation}
However, using now these with the fourth equation, we arrive to the condition
\begin{equation}
    2(\nu_1 + \nu_2 - \nu_3 - \nu_4) = 1,
\end{equation}
which is clearly unsatisfiable for integer $\nu_i\in\ZZ$.

It is worth noting that, of course, this result could have been obtained more easily from a more physical line of reasoning. Indeed, an Hamiltonian $\calH$ containing only one-qubit interactions can always be written as $\calH=h_0 I+\calH_1+\calH_2$ with $\calH_i$ containing only one-qubit terms on the $i$-th qubit. Then, $[\calH_1,\calH_2]=0$, and therefore
\begin{equation}
    e^{it\calH}=e^{it h_0}e^{it\calH_1}e^{it\calH_2}=e^{it h_0} \calU_1\otimes\calU_2,
\end{equation}
with $\calU_i$ a gate acting only on the $i$-th qubit.

Nevertheless, this example is interesting to show how the technique suggested by~\cref{eq:3conditions} can be put into action.
\end{example}
\highlight{Is there a way to do a reasoning like the above for a generic two-qubit gate?}

\section{Generators for the Toffoli gate}
Consider now the Toffoli gate, and let $\calP_{\le2}$ be the set of one- and two-qubit Pauli matrices. In this section we ask the question: \textit{\textbf{is there a time-independent $\calH\in\spanR(\calP_{\le2})$ such that $e^{i\calH}=\Toff$?}}

A canonical generator for $\Toff$ is $\calH_{\Toff}=\pi Z_1^- Z_2^- X_3^-$.
This has the nondegenerate eigenvector $\ket{1,1,-}$ with eigenvalue $-1$, which implies that any $\calH$ such that $[\calH,\calH_{\Toff}]=0$ must also stabilise $\ket{1,1,-}$.
This singles out the following $12$ generators:
\begin{equation}
\begin{gathered}
  Z_1, Z_2, X_3,\,\, Z_1 Z_2, Z_1 X_3, Z_2 X_3,\\
  X_1 (1 + X_3),\quad
  X_2 (1 + X_3),\\
  (1 + Z_1) X_2,\quad
  (1 + Z_1) Z_3,\\
  X_1 (1 + Z_2),\quad
  (1 + Z_2) Z_3,
\end{gathered}
\label{eq:toffoli_ops_after_commcondition}
\end{equation}
The first $6$ are the one- and two-qubit Pauli matrices which have $\ket{1,1,-}$ as eigenvector with nonzero eigenvalue.
The rest of the operators are built by considering an operator which annihilates this state, and then multiplying it by some remaining one-qubit Pauli matrix.
\highlight{Maybe a better way to describe this?}
We thus obtain the general form of a generator that respects the symmetries of the Toffoli. Any such $\calH$ will either preserve (first row) $\ket{1,1,-}$ or annihilate it (the other rows), which we shall write as
\begin{equation}
\begin{gathered}
  \calH =
  h_0 I + h_3^x X_3 + h_1^z Z_1 + h_2^z Z_2 \\
  + J_{13}^{zx} Z_1 X_3 + J_{23}^{zx} Z_2 X_3 + J_{12}^{zz} Z_1 Z_2 \\
  + (J_{13}^{xx} X_1 + J_{23}^{xx} X_2)(1 + X_3) \\
  + (1 + Z_1)(J_{12}^{zx} X_2 + J_{13}^{zz} Z_3)
  + (J_{12}^{xz} X_1 + J_{23}^{zz} Z_3)(1 + Z_2).
\end{gathered}
\end{equation}

Imposing directly the eigenvalue conditions on $\calH-\calH_{\Toff}$ after~\cref{eq:toffoli_ops_after_commcondition} is not doable directly, as diagonalization here would involve solving high-order polynomial equations.
Nonetheless, we can get somewhere by making some educated guesses. Let us then assume
\begin{equation}
\begin{gathered}
  h_1^z = h_2^z = -\frac{\pi}{8},\quad
  J_{12}^{xz} = J_{12}^{zx} = 0,
  \quad J_{13}^{zx} = J_{23}^{zx} = \frac{\pi}{8}, \\
  J_{13}^{xx} = J_{23}^{xx},\quad
  J_{23}^{zz} = -J_{13}^{zz}.
\end{gathered}
\end{equation}
With these assumptions, we get the following simplified expression
\begin{equation}
\begin{aligned}
  \calH &= h_0 I + h_3^x X_3 - \pi/8 ( Z_1 +  Z_2)(1 - X_3)
  % + \pi/8 (Z_1 + Z_2)X_3
  + J_{12}^{zz} Z_1 Z_2 \\
  &+ J_{13}^{xx} (X_1 + X_2)(1 + X_3)
  + J_{13}^{zz} (1 + Z_3) Z_3 + J_{23}^{zz} (1 + Z_2) Z_3.
\end{aligned}
\end{equation}
If we finally impose $J_{13}^{xx}$, so that the generator is diagonal on the first two qubits, we get
\begin{equation}
\begin{aligned}
	\HPrimeToff{} =
		&\pi/8\, Z_1 Z_2 X_3 + (h_0 - \pi/8)I +\\
		&(h_3^x + \pi/8) X_3 + (J_{12}^{zz} - \pi/8) Z_1 Z_2 +\\
		&J_{13}^{zz} (Z_1 - Z_2)Z_3.
\end{aligned}
\label{SM:eq:toffoli_HPrime_reduced}
\end{equation}
With the above simplified expression it is now possible to directly solve the eigenvalue problem.
This results in the following family of solutions:
\begin{equation}
\begin{aligned}
	% \frac{8}{\pi}\tilde{\mathcal H}_{\on{Toff}}(\nu_1, \nu_2, \nu_3, \nu_4) =
	&\tilde{\mathcal H}_{\on{Toff}} =
	\frac{\pi}{8} \bigg[
	1 + 4 \left(\nu_1 + \nu_2 + 2\nu_3 + \sqrt{(\nu_3-\nu_4)^2} \right) \\
	&- (Z_1 + Z_2)(1 - X_3) + X_3 (-2 - 8\nu_1 + 8\nu_2) \\
	% &+ (Z_1 + Z_2) X_3 \\
	&+ 4 Z_1 Z_2 \left(1/4 + \nu_1 + \nu_2 - 2\nu_3 - \sqrt{(\nu_3-\nu_4)^2} \right) \\
	&+ (Z_2 - Z_1) Z_3 \,\,\sqrt{c(\nu_1, \nu_2, \nu_3, \nu_4)}
	\bigg],
\end{aligned}
\label{eq:toff_tilde_general_solution}
\end{equation}
with
\begin{equation}
\begin{split}
	c(\nu_1, \nu_2, \nu_3, \nu_4) =
		&-(1 + 4\nu_1 - 4\nu_2 + 4\nu_3 - 4\nu_4)\\
		&\times 	(1 + 4\nu_1 - 4\nu_2 - 4\nu_3 + 4\nu_4) =\\
		&\!\!= -[(1+4\nu_{12})^2-(4\nu_{34})^2],
\end{split}
\end{equation}
for all integer values of $\nu_i$ such that $c(\nu_1, \nu_2, \nu_3, \nu_4) \ge 0$.
The corresponding spectrum of $\HPrimeToff = \HTildeToff - \HToff$ is
\begin{equation}
\begin{aligned}
	\lambda_1 &= \lambda_2 = 2\pi \nu_1, \\
	\lambda_3 &= \lambda_4 = 2\pi \nu_2, \\
	\lambda_5 &= \lambda_6 = 2\pi \nu_3, \\
	\lambda_7 &= \lambda_8 = 2\pi (\nu_3 + \lvert\nu_3 - \nu_4\rvert), 
\end{aligned}
\end{equation}
while the spectrum of $\HTildeToff$ changes only in that
$\lambda_2 = 2\pi(\nu_1 + 1/2)$.
Consistently with this, $\lambda_2$ is also the eigenvalue corresponding to the non-degenerate eigenspace of $\HToff$, while all the other eigenvalues correspond to eigenvectors orthogonal to this one.
More specifically, we have
\begin{equation}
\begin{aligned}
	\ket{\lambda_1} &= \ket{0, 0, -}, \\
	\ket{\lambda_2} &= \ket{1, 1, -}, \\
	\ket{\lambda_3} &= \ket{1, 1, +}, \\
	\ket{\lambda_4} &= \ket{0, 0, +}, \\
	\ket{\lambda_5} &= \ket{1, 0} \otimes N_5\left[(a - b) \ket0 + \ket1 \right], \\
	\ket{\lambda_6} &= \ket{0, 1} \otimes N_6\left[(a + b) \ket0 + \ket1 \right], \\
	\ket{\lambda_7} &= \ket{1, 0} \otimes N_6\left[(a + b) \ket0 - \ket1 \right], \\
	\ket{\lambda_8} &= \ket{0, 1} \otimes N_5\left[(a - b) \ket0 - \ket1 \right],
\end{aligned}
\label{eq:toffoli_eigenvectors_solution}
\end{equation}
where
\begin{equation}
\newcommand{\denom}{1 + \bar\nu_{12}}
\begin{gathered}
	a = \frac{\lvert\bar\nu_{34}\rvert}{\denom{}},
	\qquad
	b = \frac{\sqrt{\bar\nu_{34}^2 - (\bar\nu_{12} + 1)^2}}{\denom{}},
\end{gathered}
\end{equation}
% \begin{equation}
% \begin{gathered}
% 	a = 4\lvert \nu_3 - \nu_4 \rvert / c,
% 	\qquad
% 	b = \sqrt{-b_1 b_2} / c, \\
% 	% a &= \frac{4\lvert \nu_3 - \nu_4 \rvert}{1 + 4\nu_1 - 4\nu_2}, \\
% 	% b &= \frac{\sqrt{-b_1 b_2}}{1 + 4\nu_1 - 4\nu_2}, \\
% 	b_1 = (1 + 4(\nu_1 - \nu_2 + \nu_3 - \nu_4)), \\
% 	b_2 = (1 + 4(\nu_1 - \nu_2 - \nu_3 + \nu_4)), \\
% 	c = 1 + 4\nu_1 - 4\nu_2, \\
% 	N_5^2 = ( (a-b)^2 + 1 )^{-1},\quad
% 	N_6^2 = ( (a+b)^2 + 1 )^{-1}, \\
% 	N_5^2 + N_6^2 = 1.
% \end{gathered}
% \end{equation}
It is worth noting that the orthogonality of these eigenvectors follows from the easily verified property of the above coefficients: $a^2 - b^2 = 1$.
Furthermore, we note that $c(\nu_1, \nu_2, \nu_3, \nu_4) \ge 0$ cannot be satisfied unless $\nu_3 \neq \nu_4$.
This in turn, looking at \cref{eq:toffoli_eigenvectors_solution},
reveals that all the solutions are made possible by a non-trivial lifting of the degeneracy of the subspaces $\ketbra{0,1}$ and $\ketbra{1,0}$.
Let us now try to understand how and why the derived $\HPrimeToff{}$ works.
Let us use the notation $P_i \equiv \ketbra{\lambda_i}$,
and consider the projector over the last two eigenvectors.
Highlighting the 3-qubit terms, we find
\begin{equation}
\begin{aligned}
	P_7 \simeq - N_6^2 \frac{Z_1 Z_2}{4} \bigg[
		\big((a + b)^2 - 1\big) \frac{Z_3}{2}
		- (a + b) X_3
	\bigg], \\
	P_8 \simeq - N_5^2 \frac{Z_1 Z_2}{4} \bigg[
		\big((a - b)^2 - 1\big) \frac{Z_3}{2}
		- (a - b) X_3
	\bigg].
\end{aligned}
\end{equation}
The term in the Hamiltonian to which these two projectors contribute is
$2\pi \nu_{3,4} (P_7 + P_8)$, with $\nu_{3,4} = \nu_3 + \lvert \nu_3 - \nu_4\rvert$.
A little algebra reveals that the 3-qubit terms in $P_7 + P_8$ are
\begin{equation}
\begin{aligned}
	- \frac{Z_1 Z_2 Z_3}{8} \left[
		N_6^2\big((a+b)^2 - 1\big)
		+ N_5^2 \big((a-b)^2 - 1\big)
	\right] \\
	+ \frac{Z_1 Z_2 X_3}{4} \left[
		N_6^2(a+b) + N_5^2(a-b)
	\right].
\end{aligned}
\end{equation}
Recalling the definitions of $a,b,N_5,N_6$, we see that the coefficient of $Z_1 Z_2 Z_3$ vanishes, and the resulting expression becomes
\begin{equation}
	P_7 + P_8 = (...) +
	Z_1 Z_2 X_3 \frac{1 + 4(\nu_1 - \nu_2)}{16\lvert \nu_3 - \nu_4\rvert}.
\end{equation}
Substitution of the appropriate values of $\nu_i$ shows that the above term can be used to generate the 3-qubit factor $\pi/8 \,\,Z_1 Z_2 X_3$,
\emph{without introducing additional 3-qubit factors}.
In Box~\ref{tcolorbox:toffoli} are given the full expressions for the projectors and the found solutions for the Toffoli gate.
It is also interesting to note that all of the above still holds if the $X_i$ operators are replaced with $Y_i$ operators.
That is, the expressions found solving for the Toffoli, by simple substitution $X_i \to Y_i$,
also give a generator with only 2-qubit interactions for the CCY gate
(that is, the gate that applies $Y$ to the third qubit conditionally to the first 2 qubits being in the $\ket1$ state).

\begin{tbox}[label=tcolorbox:toffoli]{Toffoli}
	\begin{equation*}
	\begin{aligned}
		P_1 &= Z_1^+ Z_2^+ X_3^-,
		\qquad
		P_2 &= Z_1^- Z_2^- X_3^-,
		\qquad
		P_3 &= Z_1^+ Z_2^+ X_3^+,
		\qquad
		P_4 &= Z_1^- Z_2^- X_3^+,
	\end{aligned}
	\end{equation*}
	%
	\begin{equation*}
	\begin{aligned}
		P_5 &= Z_1^- Z_2^+ \frac{1}{2\lvert\bar{\nu}_{34}\rvert}
		\left[
			\lvert\bar\nu_{34}\rvert +
			(1 + \bar\nu_{12}) X_3 -
			\sqrt{-(1 + \bar\nu_{12})^2 + \bar\nu_{34}^2} Z_3
		\right], \\
		P_6 &= Z_1^+ Z_2^- \frac{1}{2\lvert\bar{\nu}_{34}\rvert}
		\left[
			\lvert\bar\nu_{34}\rvert +
			(1 + \bar\nu_{12}) X_3 +
			\sqrt{-(1 + \bar\nu_{12})^2 + \bar\nu_{34}^2} Z_3
		\right], \\
		P_7 &= Z_1^- Z_2^+ \frac{1}{2\lvert\bar{\nu}_{34}\rvert}
		\left[
			\lvert\bar\nu_{34}\rvert -
			(1 + \bar\nu_{12}) X_3 +
			\sqrt{-(1 + \bar\nu_{12})^2 + \bar\nu_{34}^2} Z_3
		\right], \\
		P_8 &= Z_1^+ Z_2^- \frac{1}{2\lvert\bar{\nu}_{34}\rvert}
		\left[
			\lvert\bar\nu_{34}\rvert -
			(1 + \bar\nu_{12}) X_3 -
			\sqrt{-(1 + \bar\nu_{12})^2 + \bar\nu_{34}^2} Z_3
		\right].
	\end{aligned}
	\end{equation*}
	%
	\begin{equation*}
		P_1 + P_2 = \frac{1}{4} (1 + Z_1 Z_2) (1 - X_3),
		\qquad
		P_3 + P_4 = \frac{1}{4} (1 + Z_1 Z_2) (1 + X_3).
	\end{equation*}
	%
	\begin{align*}
		P_5 + P_6 = \frac{1}{4\lvert\bar\nu_{34}\rvert} \left[
			(1 - Z_1 Z_2) \lvert\bar\nu_{34}\rvert +
			(1 - Z_1 Z_2) X_3 (1 + \bar\nu_{12}) +
			(Z_1 - Z_2)Z_3 \sqrt{\bar\nu_{34}^2 - (1 + \bar\nu_{12})^2}
		\right], \\
		P_7 + P_8 = \frac{1}{4\lvert\bar\nu_{34}\rvert} \left[
			(1 - Z_1 Z_2) \lvert\bar\nu_{34}\rvert -
			(1 - Z_1 Z_2) X_3 (1 + \bar\nu_{12}) -
			(Z_1 - Z_2)Z_3 \sqrt{\bar\nu_{34}^2 - (1 + \bar\nu_{12})^2}
		\right].
	\end{align*}
	% \begin{align*} 
	% 	P_5 + P_6 &= \frac{1}{16} \left[
	% 	4(1 - Z_1 Z_2)
	% 	+ \frac{1 + 4(\nu_1 - \nu_2)}{\lvert \nu_3 - \nu_4 \rvert}
	% 		(1 - Z_1 Z_2)X_3
	% 	- (Z_2 - Z_1)Z_3 \sqrt{16 - \frac{(1 + 4(\nu_1 - \nu_2))^2}{(\nu_3-\nu_4)^2}}
	% 	\right], \\
	% 	P_7 + P_8 &= \frac{1}{16} \left[
	% 	4(1 - Z_1 Z_2)
	% 	- \frac{1 + 4(\nu_1 - \nu_2)}{\lvert \nu_3 - \nu_4 \rvert}
	% 		(1 - Z_1 Z_2)X_3
	% 	+ (Z_2 - Z_1)Z_3 \sqrt{16 - \frac{(1 + 4(\nu_1 - \nu_2))^2}{(\nu_3-\nu_4)^2}}
	% 	\right].
	% \end{align*}
	It is easily verified from the above that
	\begin{equation*}
		P_1 + P_2 + P_3 + P_4 = \frac{1}{2} (1 + Z_1 Z_2),
		\qquad
		P_5 + P_6 + P_7 + P_8 = \frac{1}{2} (1 - Z_1 Z_2),
	\end{equation*}
	so that the sum of the projectors gives the identity as it should.
	On the other hand, multiplying by the appropriate $\nu_i$ factors, we get
	\begin{equation*}
	\begin{aligned}
		2\pi\left[\nu_1(P_1 + P_2) + \nu_2(P_3 + P_4)\right]
		&= (...) + \frac{\pi}{2} (\nu_2 - \nu_1) Z_1 Z_2 X_3, \\
		%
		2\pi\left[\nu_3(P_5 + P_6) + \nu_4(P_7 + P_8)\right]
		&= (...) + \frac{\pi}{2} (\nu_1 - \nu_2) Z_1 Z_2 X_3 + \frac{\pi}{8} Z_1 Z_2 X_3,
	\end{aligned}
	\end{equation*}
	with the last identity holding for $\nu_3\neq \nu_4$.
\end{tbox}

A different way to understand $\HTildeToff$ is to analyse the four two-dimensional subspaces on the main diagonal, exploiting the fact that $\HTildeToff{}$ acts diagonally on the first two qubits.
Straightforward calculations lead to
\begin{equation*}
\begin{aligned}
	\mel{00}{\HTildeToff}{00} &= \pi \left[(\nu_1+\nu_2)-(\nu_1-\nu_2)X\right], \\
	\mel{01}{\HTildeToff}{01} &= 2\pi\nu_3 + \pi\lvert\nu_{34}\rvert(1 - \sigma_{01}), \\
	\mel{10}{\HTildeToff}{10} &= 2\pi\nu_3 + \pi\lvert\nu_{34}\rvert(1 - \sigma_{10}), \\
	\mel{11}{\HTildeToff}{11} &= \frac{\pi}{2}\left[
		(1+2(\nu_1+\nu_2))-(1+2\nu_{12})X
	\right], \\
\end{aligned}
\end{equation*}
where
\begin{equation}
\begin{aligned}
	\sigma_{01}&\equiv\frac{(1+4\nu_{12})X + \sqrt{c}Z}{4\lvert\nu_{34}\rvert}, \\
	\sigma_{10}&\equiv\frac{(1+4\nu_{12})X - \sqrt{c}Z}{4\lvert\nu_{34}\rvert}.
\end{aligned}
\end{equation}
It can be verified that for all values of $\nu_1, \nu_2, \nu_3, \nu_4$, the two-dimensional identity and $X$ are correctly generated in the $\ket{00}$ and $\ket{11}$ spaces, respectively.
On the other hand, in the $\ket{01}$ and $\ket{10}$ spaces, the two-dimensional identity is generated as long as $\nu_3\neq\nu_4$, as was also derived before.

In particular, the class of solutions given by $\nu_1 = \nu_2 = \nu_3 = 0$ is
\begin{equation}
\begin{split}
	\frac{\pi}{8} \bigg[
	1 + 4\lvert\nu_4\rvert
	- 2 X_3 - Z_1 - Z_2
	+ (Z_1 + Z_2) X_3 \\
	+ Z_1 Z_2 (1 - 4\lvert\nu_4 \rvert)
	+ (Z_2 - Z_1) Z_3 \sqrt{16\nu_4^2 - 1}
	\bigg],
\end{split}
\label{SM:eq:toffoli_generator_nu4}
\end{equation}
for all $\nu_4\neq 0$.
It is interesting to look at the explicit form of the exponentials
generated by this class generators.
Computing $\exp(i \tilde{\mathcal H} t)$ with $\tilde{\mathcal H}$ as in \cref{SM:eq:toffoli_generator_nu4}, we get the following unitary:
\begin{equation}
	\begin{pmatrix}
		I_2 & \mathbb0 & \mathbb0 & \mathbb0 \\
		\mathbb0 & S(t, \nu_4) & \mathbb0 & \mathbb0 \\
		\mathbb0 & \mathbb0 & S(t, \nu_4) & \mathbb0 \\
		\mathbb0 & \mathbb0 & \mathbb0 & X(t)
	\end{pmatrix},
\end{equation}
where
\begin{equation}
	S(t, \nu_4) = \begin{pmatrix}
		a + b & c \\
		c & a - b
	\end{pmatrix},
\end{equation}
\begin{equation}
\begin{gathered}
	a = \frac{1 + e^{2i\pi t \nu_4}}{2},
	\qquad c = \frac{1 - e^{2i\pi t \nu_4}}{8\nu_4},  \\
	b = \frac{(-1 + e^{2i\pi t \nu_4})\sqrt{16\nu_4^2 - 1}}{8\nu_4},
\end{gathered}
\end{equation}
and
\begin{equation}
	X(t) = \frac{1}{2} \begin{pmatrix}
		(1 + e^{i\pi t}) & (1 - e^{i\pi t}) \\
		(1 - e^{i\pi t}) & (1 + e^{i\pi t})
	\end{pmatrix}
\end{equation}
For large (in modulus) values of $\nu_4$,
$a + b \to e^{2i\pi t \nu_4}$, $a - b \to 1$ and $c\to0$,
so that the exponential becomes
\begin{equation}
	\begin{pmatrix}
		I & & & & &\\
		& e^{2i\pi t \nu_4} & & & & \\
		& & 1 & & & \\
		& & & e^{2i\pi t \nu_4} & & \\
		& & & & 1 & \\
		& & & & & X(t)
	\end{pmatrix},
\end{equation}
which very closely resembles the matrix obtained by exponentiating the principal generator
$\mathcal H_{\on{Toff}} = \pi Z_1^- Z_2^- X_3^-$:
\begin{equation}
	\exp(i t \mathcal H_{\on{Toff}}) =
	\begin{pmatrix}
		I & & & \\
		& I & & \\
		& & I & \\
		& & & X(t)
	\end{pmatrix}.
\end{equation}
A different solution derived from~\cref{eq:toff_tilde_general_solution} is
\begin{equation}
\begin{aligned}
	\tilde{\mathcal H}_{\on{Toff}} =
	\frac{9\pi}{8} + \frac{3\pi}{4} X_3 - \frac{\pi}{8} (Z_1 + Z_2)
	+ \frac{\pi}{8} Z_1 Z_2 \\
	+ \frac{\pi}{8} (Z_1 + Z_2) X_3
	- \frac{\sqrt7 \pi}{8} (Z_1 - Z_2) Z_3.
\end{aligned}
\end{equation}
Moreover, it is worth noting that~\cref{eq:toff_tilde_general_solution} is only one possible family of solutions, and that different assumptions will lead to different ones.
For example, a similar reasoning as above, starting however from the assumptions $J_{23}^{zz}=J_{13}^{zz}$ will lead to solutions such as (note the use of $(Z_1+Z_2)$ terms here, making this solution not derivable from~\cref{eq:toff_tilde_general_solution}):
\begin{equation}
\begin{aligned}
	\HTildeToff{} =\,
		&\frac{9\pi}{8} - \frac{7\pi}{8}X_3 + \frac{\sqrt{15}\pi}{8} Z_3 + \frac{\pi}{8}Z_1 Z_2 \\
		&\frac{\pi}{8}(Z_1+Z_2)\left(-1+\frac{5}{2}X_3 + \frac{\sqrt{15}}{2}Z_3\right).
\end{aligned}
\end{equation}


\section{ML to the rescue}
\highlight{Probably these sections will be merged at some point, they're a bit to dispersive.}

What kind of ML do we use here? Does what we do even qualify as ML? Why do we use supervised learning instead of other types of learning?


\section{Analytical results}

\subsection{Applications to Toffoli and Fredkin}

\section{Numerical results}

\subsection{Toffoli and Fredkin}

\subsection{Other cool gates}

\subsection{Analytical results}

\subsection{Numerical approximate results in constrained scenarios}

% \EnableTOCUpdates
