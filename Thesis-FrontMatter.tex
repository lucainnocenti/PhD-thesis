%!TEX root = ./Thesis.tex

% Use Roman numerals (i, ii, iii, etc.) for page numbers in the front matter.
\pagenumbering{roman}

%%%%%%%%%%%%%%%%%%%%%%%%%%%%%%%%%%%%%%%%%%%%%%%%%%%%%%%%%%%%%%%%%
%% TITLE PAGE.
%%%%%%%%%%%%%%%%%%%%%%%%%%%%%%%%%%%%%%%%%%%%%%%%%%%%%%%%%%%%%%%%%

% No headers or footers on the title page.
\thispagestyle{empty}

\begingroup
\centering
\setstretch{1.0}
~
\\[1em]
\sffamily\bfseries\fontsize{26}{31.2}\selectfont
\DocumentTitle
% \\
% Use Manual Line Breaks If Necessary
\\[0.4in]
\normalfont\large
Thesis by
\\[0.25em]
\sffamily\bfseries\Large
\AuthorName
\\[0.4in]
\normalfont\normalsize
In Partial Fulfillment of the Requirements
\\[0.5em]
for the Degree of
\\[0.5em]
Doctor of Philosophy
\\[0.5em]
in
\\[0.5em]
Physics
\vfill
% \includegraphics[height=1.8in]
% {Figure-SchoolLogo}
% \\[1.5em]
Queen's University Belfast
\\[0.5em]
Belfast, UK
\\[1.5em]
2020
\\[0.5em]
(Defended \emph{\textbf{who knows when}})
\par
\endgroup

\clearpage

%%%%%%%%%%%%%%%%%%%%%%%%%%%%%%%%%%%%%%%%%%%%%%%%%%%%%%%%%%%%%%%%%
%% COPYRIGHT PAGE.
%%%%%%%%%%%%%%%%%%%%%%%%%%%%%%%%%%%%%%%%%%%%%%%%%%%%%%%%%%%%%%%%%

\pagestyle{plain}
\setcounter{page}{2}

\begingroup
\centering
\setstretch{1.0}
\null
\vfill
{\sffamily\textcopyright}~2016
\\[0.5em]
\AuthorName
\\[0.5em]
All Rights Reserved
\par
\endgroup

\clearpage

%%%%%%%%%%%%%%%%%%%%%%%%%%%%%%%%%%%%%%%%%%%%%%%%%%%%%%%%%%%%%%%%%
%% DEDICATION PAGE.
%%%%%%%%%%%%%%%%%%%%%%%%%%%%%%%%%%%%%%%%%%%%%%%%%%%%%%%%%%%%%%%%%

% \begingroup
% \centering
% \setstretch{1.0}
% ~
% \\[1in]
% \textit{Insert dedication here}
% \par
% \endgroup

\clearpage

%%%%%%%%%%%%%%%%%%%%%%%%%%%%%%%%%%%%%%%%%%%%%%%%%%%%%%%%%%%%%%%%%
%% ACKNOWLEDGMENTS.
%%%%%%%%%%%%%%%%%%%%%%%%%%%%%%%%%%%%%%%%%%%%%%%%%%%%%%%%%%%%%%%%%

\chapter*{Acknowledgments}
\addcontentsline{toc}{chapter}{Acknowledgments}

\clearpage

%%%%%%%%%%%%%%%%%%%%%%%%%%%%%%%%%%%%%%%%%%%%%%%%%%%%%%%%%%%%%%%%%
%% ABSTRACT.
%%%%%%%%%%%%%%%%%%%%%%%%%%%%%%%%%%%%%%%%%%%%%%%%%%%%%%%%%%%%%%%%%

\chapter*{Abstract}
\addcontentsline{toc}{chapter}{Abstract}

In this Thesis, we discuss applications of machine learning to quantum information science.
The interface between these two fields has been the subject of much research, recently, driven by the many successes of machine learning for diverse pattern recognition tasks. The work reported in this Thesis addresses precisely the potential of such an interdisciplinary line of research and illustrates how machine learning provides a valuable add-on to standard techniques used in the context of quantum technologies.

Specifically, we will first study the problem of devising time-independent dynamics implementing target quantum operations.
This involves a challenging optimisation task, which is hard to tackle with standard numerical tools.
We demonstrate how the use of supervised learning algorithms can dramatically speed-up this task.

We then consider the tasks of engineering and characterising quantum states in large Hilbert spaces. Motivated by strong experimental reasons, we consider explicitly the embodiment of such multi-dimensional quantum systems provided by the orbital angular momentum and polarisation degrees of freedom of light.
We devise a protocol involving quantum walks to implement arbitrary states by only making use of the possibility to couple polarisation and orbital angular momentum through relatively recent technological advances in the field of linmear optics.
This protocol relies solely on the properties of quantum walk dynamics, and is therefore applicable to different types of architectures.

We then present an experimental demonstration of this engineering strategy, and consider the issue of assessing the quality of the states thus generated.
We show how different machine learning algorithms, including supervised and unsupervised learning ones, are able to tackle this problem and provide useful information in realistic experimental conditions.

% \chapter*{Note}
% {\HUGE Fuck you}
% {\Huge Note:}
% \vfill
% {\large\bfseries Note:}
% To provide further structure to the thesis, paragraph headings are used to give the reader a quick idea of what that is addressed in that section of text \highlight{(ugh)}.


\clearpage

%%%%%%%%%%%%%%%%%%%%%%%%%%%%%%%%%%%%%%%%%%%%%%%%%%%%%%%%%%%%%%%%%
%% TABLE OF CONTENTS (TOC), LISTS OF FIGURES, TABLES, ETC.
%%%%%%%%%%%%%%%%%%%%%%%%%%%%%%%%%%%%%%%%%%%%%%%%%%%%%%%%%%%%%%%%%

\tableofcontents

% \listoffigures

% \listoftables

\clearpage

% Use Arabic numerals (1, 2, 3, etc.) for subsequent page numbers.
\pagenumbering{arabic}
