\usepackage{amsmath,amssymb}
\usepackage{mathtools}

\usepackage{blindtext} % Package to generate dummy text throughout this template 

%%%%%%%%%%%%%%%%%% FONT CHOICE %%%%%%%%%%%%%%%%%%%%%%

% \usepackage{unicode-math}
\usepackage[bold-style=ISO]{unicode-math}
\usepackage{parskip}
\defaultfontfeatures{Scale=MatchLowercase}
\setmainfont{TeX Gyre Pagella}[Scale = 1.0]

% \setmathfont{Pagella Math}
% \setmathfont{XITS Math}
\usepackage{bm}

% \renewcommand{\boldsymbol}{\symbf} % this is because unicode-math with the bold-style option wants you to use \symbf rather that \boldsymbol, for some reason

% \linespread{1.05} % Line spacing - Palatino needs more space between lines

%%%%%%%%%%%%%%%%%% END FONT CHOICE %%%%%%%%%%%%%%%%%%%%%%

\usepackage{xspace} % to avoid newcommand eating trailing space
\usepackage{etoolbox}
\usepackage{microtype} % Slightly tweak font spacing for aesthetics

\usepackage[english]{babel} % Language hyphenation and typographical rules

\usepackage[hmarginratio=1:1,top=32mm,columnsep=20pt]{geometry} % Document margins
\usepackage{setspace} % provides \setstretch
\usepackage[small,labelfont=bf,up,textfont=it,up]{caption} % Custom captions under/above floats in tables or figures
\usepackage{booktabs} % Horizontal rules in tables
\usepackage{fullpage}

\usepackage{lettrine} % The lettrine is the first enlarged letter at the beginning of the text

\usepackage{enumitem} % Customized lists
\setlist[itemize]{noitemsep} % Make itemize lists more compact

\usepackage{acronym}
\newacro{AD}{Automatic Differentiation}
\newacro{ML}{Machine Learning}
\newacro{NN}{Neural Network}
\newacro{SGD}{Stochastic Gradient Descend}

\newacro{QW}{Quantum Walk}
\newacro{DTQW}{Discrete-Time Quantum Walk}
\newacro{CTQW}{Continuous-Time Quantum Walk}

\newacro{SLM}{Spatial Light Modulator}
\newacro{SAM}{Spin Angular Momentum}
\newacro{OAM}{Orbital Angular Momentum}

\usepackage{todonotes}

\usepackage{abstract} % Allows abstract customization
\renewcommand{\abstractnamefont}{\normalfont\bfseries} % Set the "Abstract" text to bold
\renewcommand{\abstracttextfont}{\normalfont\small\itshape} % Set the abstract itself to small italic text

% \usepackage{titlesec} % Allows customization of titles
% \renewcommand\thesection{\Roman{section}} % Roman numerals for the sections
% \renewcommand\thesubsection{\roman{subsection}} % roman numerals for subsections
% \titleformat{\section}[block]{\large\scshape\centering}{\thesection.}{1em}{} % Change the look of the section titles
% \titleformat{\subsection}[block]{\large}{\thesubsection.}{1em}{} % Change the look of the section titles

% \usepackage{fancyhdr} % Headers and footers
% \pagestyle{fancy} % All pages have headers and footers
% \fancyhead{} % Blank out the default header
% \fancyfoot{} % Blank out the default footer
% \fancyhead[C]{Running title $\bullet$ May 2016 $\bullet$ Vol. XXI, No. 1} % Custom header text
% \fancyfoot[RO,LE]{\thepage} % Custom footer text

\usepackage{titling} % Customizing the title section

\usepackage{physics}

\usepackage{csquotes}

\usepackage{easyReview}

\usepackage{graphicx}
\graphicspath{{./Figures/}{./Figures/gate-learning/}{./Figures/quantum-walks/}}
\usepackage{wrapfig}
\usepackage{subfig}

\usepackage{xcolor}
\usepackage{nameref}
\usepackage[colorlinks=true]{hyperref}
\usepackage[nameinlink]{cleveref}
\crefname{appsec}{Appendix}{Appendices}


\usepackage{tikz}

\usepackage{amsthm}
% \newtheorem{example}{Example}

\usepackage[many]{tcolorbox}
% \tcbuselibrary{theorems}
% \newtcbtheorem
%   [number within=section]% init options
%   {example}% name
%   {Example}% title
%   {%
%     breakable, enhanced,
%     colback=green!5,
%     colframe=green!35!black,
%     fonttitle=\bfseries,
%     title={Porco dio}
%   }% options
%   {example}% prefix
\newtcolorbox[
    auto counter,
    number within=section,
    crefname={example}{examples}
]{example}[1][]{
  enhanced,
  breakable,
%   colback=green!5,
%   colframe=green!35!black,
  fonttitle=\bfseries,
  title={Example \thetcbcounter},
  #1
}
\newtcolorbox[auto counter]{tbox}[2][]{%
	enhanced, float*=t,
	colback=gray!5!white, colframe=gray!75!black,
	width=\textwidth,
	title={Box \thetcbcounter: #2}, #1
}


\usepackage[
    backend=biber,
    style=alphabetic,
    maxcitenames=1,
    url=true,
    doi=true,
    sorting=ynt,
    backref=true
]{biblatex}
\DeclareNameAlias{author}{given-family}
\addbibresource{Thesis.bib}
