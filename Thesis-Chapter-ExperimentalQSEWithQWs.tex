\chapter{Experimental engineering of qudit states}

\textit{Introduction ---}
The preparation of high-dimensional quantum states is of great significance in quantum information science and technology. 
Compared to qubits, \textit{qudit} states -- describing quantum systems in $d$-dimensional spaces --
enable stronger foundational tests of quantum mechanics~\cite{Vertesi2010,brunner2014bell, lapkiewicz2011experimental} and better-performing applications in secure quantum communications~\cite{bechmannpasquinucci2000quantum, fitzi2001quantum, cerf2002security, bru2002optimal, acin2003security, langford2004measuring}, quantum emulation~\cite{Buluta2009, Neeley2009}, quantum error correction~\cite{Chuang1997,Duclos-Cianci2013,Michael2016}, fault-tolerant quantum computation~\cite{bartlett2002quantum, ralph2007efficient, Lanyon2009, campbell2012magicstate, Campbell2014}, and quantum machine learning \cite{Schuld2015, Dunjko2017, Biamonte2017}. 

Protocols performed on systems living in large Hilbert spaces require great control in light of the number of parameters required to describe states and operations. Nonetheless, qudit states have been prepared successfully in various physical settings~\cite{Leibfried1996, Hofheinz2009, Neeley2009, Anderson2015, Walborn2006, Lima2011, Rossi2009, Dada2011, Anderson2015, Heeres2017, Rosenblum2018, Chu2018}. Such schemes rely on \textit{ad hoc} strategies whose dependence on the underpinning dynamics makes their translation across different physical platforms difficult. 

A promising way to achieve a higher degree of platform-universality is the use of the rich dynamics offered by \acp{QW}~\cite{aharonov1993quantum, Kempe2003, Venegas-Andraca2012}. These can be thought of as the quantum counterparts of classical random walks and comprise -- in their discrete version -- a qudit, named \emph{walker}, endowed with an internal two-dimensional degree of freedom dubbed \emph{coin}~\cite{ambainis2001one}. \acp{QW} have been successfully implemented~\cite{Manouchehri2014} in systems as diverse as trapped atoms~\cite{cote2006quantum} and ions~\cite{Schmitz2009,Zahringer2010}, photonic circuits~\cite{Perets2008, Peruzzo2010, broome2010bdp, Schreiber2010, Rohde2011, sansoni2012quantum, boutari2016time, cardano2015quantum, cardano2016statistical, caruso2016maze}, and optical lattices~\cite{Meinert2014}. An approach for state engineering based on their dynamics offers hope of being applicable in a variety of different systems, independently of the details of the physical implementation.

While the \ac{QW} dynamics was previously shown to allow the engineering of {\it specific} walker's states~\cite{chandrashekar2008optimizing,majury2016robust}, in Ref.~\cite{Innocenti2017} a scheme was proposed to use discrete-time \acp{QW} on a line to prepare {\it arbitrary} qudit states with high probability and fidelity, in principle for arbitrary dimensions of the target qudit states.
This is achieved by enhancing the degree of control over the walk's dynamics through the arrangement of suitable step-dependent \emph{coin} operations, which affect the coin-walker quantum correlations by \emph{de facto} steering the state of the walker towards the desired final state, and finally projecting in the coin space. This removes correlations between walker and coin, thus producing a pure walker state with the desired features. 

\begin{figure*}[t!]
\includegraphics[width=\textwidth]{fig1.pdf}
\caption{Setup for the quantum state engineering toolbox. {\bf a)} Conceptual scheme of the protocol. At each step of \ac{QW} the coin operator is changed to obtain a target state in the output. {\bf b)} A single-photon source, composed of a periodically-poled potassium titanyl phosphate (PPKTP) crystal, generates pairs of photons that are coupled in a single-mode fibre (SMF). One photon acts as trigger while the other is prepared in $\ket{\psi_0}=\ket{+}\otimes 
\ket{0}$ through polarization controllers and a polarizing beam splitter (PBS). Five sets of quarter (QWP) and half (HWP) waveplates implement operators $\{ C_i\}$ for each step. Five Q-plates (QP) implement the shift operator of the \ac{QW}
$\{S_i\}$. The detection stage consists of a PBS followed by a spatial light modulator (SLM), 
a SMF and an avalanche photodiode detector (APD), for projection onto $\ket{+} \otimes \ket{\psi}$. %Displaying different holograms on SLM we are able to measure OAM components in the target state. 
{\bf c)} Pictures of OAM modes of the output states after PBS, obtained with coherent light. From right: OAM eigenstate corresponding to $m=5$; balanced superposition of $m=\pm 5$; balanced superposition of all OAM components covered by 5-step \ac{QW} $m=\{\pm 5, \pm 3, \pm 1\}$. }
\label{app}
\end{figure*}

In this paper, we use of the scheme of Ref.~\cite{Innocenti2017} to demonstrate a state-engineering protocol based on the controlled dynamics generated by \acp{QW}. We use the orbital angular momentum (OAM) degree of freedom of single-photon states as a convenient embodiment of the walker~\cite{zhang-oam-qw-2010,goyal2013implementing,cardano2015quantum}. OAM-based experiments offer the possibility to cover Hilbert spaces of large dimensions in light of the favourable (linear) scaling of the number of optical elements with the size of the walk. Moreover, the scheme allows for the full control of the coin operation that is key to the walk implementation of the walk. In order to demonstrate the versatility of our scheme, we focus on the interesting classes of cat-like states and spin-coherent states~\cite{brune_Cat_1992,monroe_Cat_1996}. Furthermore, we show experimentally the capability of engineering arbitrary states.
The quality of the generated states and the feasibility of the experimental protocol that we have put in place, demonstrate the effectiveness of a hybrid platform for quantum state engineering. Such platform holds together a programmable quantum system, the photonic \ac{QW} in the angular momentum, and classical optimization algorithms to effectively reach a given target. 

\textit{Engineering quantum walks.--}
We consider a discrete-time \ac{QW} with a two-dimensional coin with logical states labelled as $\{\ket{{\downarrow}}_c, \ket{{\uparrow}}_c\}$. The dynamics are made up of consecutive unitary steps. At step $t$, a \emph{coin operator} $\hat{\mathcal{C}_{t}}$ changes the coin state and is then followed by a \emph{shift operator} $\shiftS$, which moves the walker conditionally to the coin state. Such transformations are described by the operators
\begin{equation}
\hat{\mathcal{C}_t}=
\left(
\begin{array}{ll}
e^{i \xi_t} \cos{\theta_t} &  e^{i \zeta_t} \sin{\theta_t} \\
-e^{-i \zeta_t} \sin{\theta_t} & e^{-i \xi_t} \cos{\theta_t}
\end{array}
\right),
\label{coinExpr}
\end{equation}
which accounts for the coin tossing, and
%\begin{equation}
$\shiftS=\sum_k |k-1\rangle \langle k|_w\otimes |{\downarrow}\rangle \langle {\downarrow}|_c+ |k+1\rangle \langle k|_w\otimes |{\uparrow}\rangle \langle{\uparrow}|_c$,
%\label{shift}
%\end{equation}
which realizes the conditional motion of the walker. Here $k$ is the lattice-site occupied by the walker and $\{\theta_t,\xi_t,\zeta_t \}$ are parameters identifying a unitary transformation in two dimensions. The evolution through $n$ steps of the {QW} is given by $\hat{U}=\prod_{t=1}^n \shiftS\hat{\mathcal{C}_t}$.

\begin{figure*}[t!]
\includegraphics[width=\textwidth]{fig2}
\caption{
	Experimental results for the engineering of angular momentum cat states. {\bf a)} Representation on a Bloch-like ball of the four target states corresponding to the superposition of $\ket{\pm 5}$, which correspond corresponding to OAM states with maximum and minimum projection of the angular momentum along the quantization axis. {\bf b)} Population of the OAM components after 5-step {QW} for the states $\ket{\psi_i}~(i=1,2,3,4)$ in panel {\bf a)}. Odd-$m$ position states (bold numbers on $x$-axis) should be the only ones involved in the state engineering. However, we report also the populations of even-$m$ position states (light-black numbers on $x$-axis) to illustrate possible imperfections at generation and detection stages. The error bars associated with experimental populations are shown by the transparent areas on top of each histogram. {\bf c)}-{\bf d)} Distributions of probabilities $P_i=\langle B^{(j)}_i\vert\rho_\text{exp}\vert B^{(j)}_i\rangle~(j=1,2)$ that the experimental walker state $\rho_\text{exp}$ is found to be one of the elements of the bases $B^{(j)}=\{\ket{\psi_p},\ket{\psi_{p+1}},\ket{\pm4},\ket{\pm3},\ket{\pm2},\ket{\pm1},\ket{0}\}$ with $p=1$ for $j=1$ and $p=3$ for $j=2$. All the error bars are due to Poissonian uncertainties, propagated through Monte Carlo methods. The state fidelities ${\cal F}$ are calculated as described in the main text.
}
\label{fig55}
\end{figure*}

In Ref.~\cite{Innocenti2017} it was shown that it is always possible to find a set of coin operators $\{\hat{\mathcal{C}_t}\}_{t=1}^{n}$ that produce an arbitrary target state in the full coin-walker space. In addition, via suitable projection in the coin space, arbitrary walker states can also be obtained. The identification of the correct set of coin operators is enabled by a classical algorithm to maximize the fidelity between the final state of the walker, after projection of the coin, and the target $(n+1)$-dimensional state. It is worth noting that all states can be reachable with unit fidelities (albeit probabilistically) and high probabilities, regardless of the number of steps $n$ (see Ref.~\cite{SI}).

To demonstrate the effectiveness of this approach for state engineering of high-dimensional spaces, we here focus on classes of physically relevant states. First, we consider the synthesis of angular-momentum Schr\"odinger cat states~\cite{AMcat}, achieved by engineering coherent superpositions of extremal walker positions. The correspondence between the position space of the walker and an angular momentum of quantum number $n/2$, which will be illustrated and clarified later in this paper, makes the \ac{QW} perfectly suited to synthesize this class of states. Schr\"odinger cat states play a crucial role in the investigations on foundations of quantum mechanics~\cite{schrodingerCAT} and their generation is at the core of various quantum engineering protocols~\cite{brune_Cat_1992, monroe_Cat_1996,agarwalCAT1997, zhang2016creating}. The second class of states that we consider is \emph{spin-coherent states}~\cite{ulyanov_spin1999}, which are the spin-like counterpart of \emph{coherent states} of a quantum harmonic oscillator. Finally, in order to validate the flexibility of our approach, we demonstrate high-quality engineering of both balanced, and randomly sampled states.

\textit{Experimental apparatus ---}
We have implemented a discrete-time \ac{QW} with $n=5$ steps, using the angular momentum states of light $\{|m\rangle_w\}$ ($m=\pm 5, \pm 3, \pm 1$) as the physical embodiment of the walker, while logical states of the coin are encoded in circular-polarization states $\{|R\rangle, |L\rangle\}$. We dub such degree of freedom as {\it spin angular momentum} (SAM) to mark the difference with OAM. Our experimental setup, which is shown schematically in Fig.~\ref{app} and follows Refs.~\cite{cardano2015quantum,cardano2016statistical}, allows for the full coin-walk evolution to take place in a single light beam, thus avoiding a nonlinear growth of optical paths as in previous interferometric implementations~\cite{zhang-oam-qw-2010,goyal2013implementing,cardano2015quantum}. Such scheme guarantees a linear scaling of the number of optical elements needed to implement a $n$-step QW (see Ref.~\cite{SI}). Arbitrary coin operators are achieved through a sequence of suitably arranged and oriented quarter- and half-waveplates~\cite{simonMukunda}. The shift operator $\shiftS$ is instead implemented using a \emph{Q-plate} (QP)~\citep{marrucci-2006spin-to-orbital}, an active device that uses an inhomogeneous birefringent medium to convert SAM into OAM  and that can conditionally change the values of the OAM by a quantity $2q$ (here $q$ is the topological charge of the device) according to transformations 
\begin{equation}
\begin{aligned}
|L,m\rangle&\stackrel{\text{QP}}{\longrightarrow} \cos{ \frac{\delta}{2}}|L,m\rangle + i e^{2 i \alpha_0}\sin{\frac{\delta}{2}}|R,m+2q\rangle,\\
|R,m\rangle&\stackrel{\text{QP}}{\longrightarrow} \cos{ \frac{\delta}{2}}|R,m\rangle + i e^{-2 i \alpha_0}\sin{\frac{\delta}{2}}|L,m-2q\rangle.
\end{aligned}
\end{equation}
The additional phase $\alpha_0$ between the two polarizations is compensated by changing the orientations of the waveplates which implement the coin operator of the subsequent step.

\begin{figure*}[t]
\includegraphics[width=\textwidth]{fig3}
\caption{Experimental results for the engineering of SCSs and their coherent superposition: {\bf a)} Bloch-sphere representation for the mutually orthogonal SCSs $\ket{S_1}$ and $\ket{S_2}$.
{\bf b)} Probability distributions associated to the projection of $\ket{S_1}$ onto the computational basis. As previously explained, we also consider the contribution of even OAM components.
{\bf c)} Probability distribution corresponding to the basis that contains the target state itself $\ket{S_1}$, generated with the fidelity reported in the panel. Such orthonormal basis, {$S_i$} with $i= 1 ...6$, contains eigenstates of $\hat{S}_x$ for a particle with spin $s=5/2$ that are in turn all spin-coherent states.
{\bf d)} Experimental probability distribution on computational basis for $\ket{\psi_2}=\frac{1}{\sqrt{2}}\left (\ket{S_1}- \ket{S_2} \right)$. Only components $\{-5, -1, 3\}$, corresponding to logical states $\{1,3,5\}$, have non-zero probabilities.
{\bf e)} Quantum state fidelity evaluated measuring state $\ket{\psi_2}$ on the orthonormal basis that contains state $\ket{\psi_1}$, as described in the main text. 
{\bf f)} Summary of quantum state fidelities for the $32$ states generated in the experiment. The average fidelity, $\bar{\mathcal{F}}=0.954\pm 0.001$, is reported by the magenta area. 
}
\label{figSpin}
\end{figure*}

Single-photon states are generated via a type-II, collinear spontaneous-parametric-down-conversion source [cf. Fig.~\ref{app}]. 
The photons emitted by the source are separated with a polarizing beam splitter (PBS) and coupled to two single-mode fibers (SMF). One photon acts as the trigger signal, while the other one undergoes the \ac{QW} evolution. After the propagation in the SMF and the first PBS, the initial state of the walker and coin is prepared in $\ket{\psi_0}_{wc}=\ket{0}_w \otimes \ket{+}_c$ with $\ket{+}_c=(\ket{{\uparrow}}_c+\ket{{\downarrow}}_c)/\sqrt2$. At the end of an $n$-step \ac{QW}, the protocol involves a projection of the coin state onto $|+\rangle_c$. This is experimentally implemented by a final PBS. The OAM analysis is performed through a \emph{spatial-light modulator} (SLM) followed by coupling into a single-mode fiber, which allows for the measurement of arbitrary superposition of OAM components with high accuracy~\cite{bolduc2013holo,dambrosio_mubs2013}. The quantum state fidelity between the actual state of the walker and the target $(n+1)$-dimensional state is estimated by projecting the OAM state onto a basis that contains the given target state [cf. Fig.\ref{app}]. 

\textit{Engineering cat-like states in high dimensions.--} Our investigation on the engineering of quantum states living in Hilbert spaces of large dimensions starts from coherent superpositions of two extremal lattice sites of the walker. The isomorphism of the OAM with an angular momentum of quantum number $n/2$ allows us to put in correspondence position states of the walker on the lattice $\ket{\pm 5}$ with angular momentum states with minimum and maximum projections onto the quantization axis $\ket{\pm 5/2}$ (for simplicity of notation, we will use position states only). Such isomorphism makes a coherent superposition state such as $(\ket{5}+e^{i\varphi}\ket{-5})/\sqrt{2}$ (with $\varphi$ a suitable phase) a faithful angular momentum Schr\"odinger cat state~\cite{AMcat}, thus benchmarking the performance of our experiment with a relevant class of states~\cite{chandrashekar2008optimizing,zhang2016creating,majury2016robust} used in quantum sensing~\cite{zeilengerOAM,dambrosio_gear2013}.

In Fig.~\ref{fig55} we report experimental results for the generation of four of such states, which are conveniently pictured as states pointing towards the poles of a Bloch-like ball. Quantum coherence between the components of such states has been tested by changing their relative phase. The values of the state fidelity between experimentally synthesized states and their respective target ones are reported in Fig.~\ref{fig55}. Hereafter we compute fidelities by projecting the state on the orthonormal basis which includes the target qudit in the 6-dimensional subspace associated to our 5-step \ac{QW}, generated by OAM eigenstates $\{|m\rangle_w\}$ ($m=\pm 5, \pm 3, \pm 1$).

The second class of relevant states that we addressed are \emph{spin-coherent states} (SCSs)~\cite{agarwalCAT1997}. These are the counterparts of coherent states of the harmonic oscillator for a particle with spin $s$~\cite{radcliffeSpin,arecchiSpinCoherent,agarwalCAT1997,vedralSpin}. 
SCSs are eigenstates -- with eigenvalue $s$ -- of the component of the total spin-momentum operator $\hat{S}$ pointing along the direction identified by the polar spherical angles $\{ \theta, \phi \}$~\cite{arecchiSpinCoherent,agarwalCAT1997,ulyanov_spin1999,yenLee_spin2015} 
A decomposition of such states over the $\{\ket{s_z}\}$ basis of the projected spin along z-direction ($\hat{S}_z$) reads
\begin{equation}
\begin{aligned}
 \ket{s,\theta,\phi} &= \sum_{s_z=-s}^{s}\sqrt{\frac{(2s)!}{(s+s_z)!(s-s_z)!}} e^{-i\phi s_z} C_\theta^{s+s_z}
S_\theta^{s-s_z}\ket{s_z}
 \label{spin_coherent1}
\end{aligned}
   \end{equation}
with $C_\theta=\sqrt{1-S^2_\theta}=\cos(\theta/2)$. SCSs have various applications in condensed matter physics, in particular in quasi-exactly solvable models, for Wigner-Kirkwood expansion and in quantum correction to energy quantization rules~\cite{ulyanov_spin1999}. At the foundational level, they can be used to generate Schr{\"o}dinger cat states~~\cite{agarwalCAT1997}. 

Although SCSs are in general not orthogonal, they form a convenient basis. Moreover, as two SCSs pointing in opposite azimuthal directions are orthogonal for $\theta\sim\pi/2$, by restricting the attention to $\{\ket{s,\pi/2,\phi},\ket{s,-\pi/2,\phi}\}$ we would be dealing with an orthonormal basis, which we can use to construct the analogous of a Bloch ball for a two-level system (cf. Fig.~\ref{figSpin}a). We have thus engineered $\ket{S_1}\equiv\ket{5/2,\pi/2,0}$ and $\ket{S_2}\equiv\ket{5/2,-\pi/2,0}$, and considered the experimental synthesis of balanced coherent superpositions of such states.
Furthermore, $S_1$ and $S_2$ are also eigenstates of $\hat{S}_x$ operator. In Fig.~\ref{figSpin}b-c the state $S_1$ is projected firstly on the computational basis, the eigenstates of $\hat{S_z}$, and then on the basis $\{S_i\}$, with $i=1...6$, which consists of $\hat{S}_x$ eigenstates. Balanced superpositions of $S_1$ and $S_2$ are akin to the Schr\"odinger cat states built on coherent states of a harmonic oscillator, as they exhibit signatures of non-classical interference~\cite{agarwalCAT1997,SI}. For instance, only even (odd) components of the logical basis enter the superposition $\ket{S_1}+\ket{S_2}$ ($\ket{S_1}-\ket{S_2}$), a parity rule that is fully analogous to the one characterizing even (odd) bosonic cat states. Thanks to the isomorphism between the spaces of OAM and of arbitrary angular momentum equal to $n/2$, we can generate SCS mapping the basis $\{ \ket{s_z}\}$ in (\ref{spin_coherent1}) into the basis of the \ac{QW} $\{|m\rangle_w\}$. The results are illustrated in Fig.~\ref{figSpin}a-e, where we show the high quality of both the generated SCSs and SCS-based cat states. %In particular, in Fig.\ref{spin_coherent1}.d we observe the odd parity that characterizes the superposition with minus sign.

\emph{Engineering arbitrary qudits.--} In order to demonstrate the flexibility of our scheme, we have addressed the generation of states of arbitrary complexity, starting from balanced states and then moving towards randomly chosen states. Balanced states are challenging as one needs to ensure equal population of all their components, a condition that is very prone to experimental imperfections. Assessing the quality of generation of such states provides a significant benchmark to the effectiveness of the procedure. We have then engineered the element of a Fourier basis associated to the Hilbert space of the walker. This choice is motivated by the importance of quantum Fourier transform in quantum algorithms \cite{chuang2010}, as well as its role in identifying mutually unbiased bases for quantum cryptography and communication in high-dimensions~\cite{durtMUB,VatanMUBS,brierley_mubs2013,dambrosio_mubs2013}.

Final measurements concern the generation of randomly-chosen qudits. We have engineered up to 5 states with real-valued amplitudes and 5 with complex-valued ones, where the state components are sampled from a uniform distribution (cf. Ref.~\cite{SI}).
In Fig.~\ref{figSpin}f quantum state fidelities are reported for all experimental engineered states, including the Fourier basis and randomly sampled qudits, where the red area shows the average fidelity and its uncertainty ($\mathcal{F}{=}0.954 \pm 0.001$)~\cite{SI}. Such test provides a further proof of the effectiveness of the strategy demonstrated in our experiment.
%\begin{figure}[t]
%\includegraphics[width=\columnwidth]{fig4.pdf}
%\caption{Summary of quantum state fidelities for the $32$ states generated in the experiment. The average fidelity, $\bar{\mathcal{F}}=0.954\pm 0.001$, is reported by the magenta area. 
%}
%\label{histo}
%\end{figure} 

\textit{Discussion.--} We have successfully tested a QW-based quantum state engineering strategy assisted by numerical optimization~\cite{Innocenti2017}. Our tests were performed in a photonic platform using OAM as the embodiment of a quantum walker. This allowed us to implement a five-step QW, without nonlinear overhead in the number of required optical paths and with full control on the preparation, coin-operation, and detection stages. We showed the effectiveness of the protocol, demonstrating its ability to synthesize high-quality cat-like states. Our results reinforce the idea that numerical optimization complementing a complex QW dynamics is effective for high-dimensional state engineering. A natural generalization of this novel paradigm could be the engineering in the multipartite scenario, exploiting quantum correlations between multiple walkers. Regarding the research of the coin, further improvements of our approach can be envisaged by identifying appropriate routines to optimize the state engineering process in the presence of actual experimental imperfections. To this end, machine learning algorithms can be a promising add-on to our numerical optimization approach to adapt the coin operators to a given experimental implementation.
