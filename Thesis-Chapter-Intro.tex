\chapter{Introduction}

Quantum information science studies how information behaves and can be manipulate at the quantum level.
The rationale behind studying such a thing is not just in the fundamental understanding that can be gained froma theoretical point of view. Indeed, ever since Feynman first introduced the idea of \emph{exploiting} the peculiarities of quantum mechanics to solve problems harder to reach within the realm of classical computation~\cite{feynman1982simulating}, more and more research has been devoted into trying to understand exactly why and how quantum mechanics can provide advantages for information processing.
In fact, the realisation itself that this feat was indeed possible has prompted a stark interest into a broad variety of research directions, for which the term \emph{quantum revolution} was coined~\cite{dowling2003quantum}.

Among the many approaches explored in the course of the last century, the most notable areas of research fall, broadly speaking, under the names of quantum computation~\cite{shor1997polynomial}, quantum simulation~\cite{lloyd1996universal}, quantum communication~\cite{bennett1993teleporting}, and quantum cryptography~\cite{bennett2014quantum}.
\highlight{(second quantum revolution stuff?)}

Focusing on quantum computation.


\section{Quantum information}

\section{Machine Learning}

\section{Quantum walks?}
