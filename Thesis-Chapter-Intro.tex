\chapter{Introduction}
Quantum information science~\cite{nielsen2006quantum,watrous2018theory} studies how information behaves and can be manipulated at the quantum level.
The reason for studying this is manyfold. One the one hand, quantum mechanics challenged our previously naive way of understanding how information behaves, telling us that the standard rules of probability theory cease to apply in some realms of the real world.
On the other hand, from a more practical point of view, there is much to gain from such a deeper understanding of how information can be manipulated. Indeed, so much of the modern age relying on the capability to manipulate ever vaster heaps of data, it is only natural that an insight telling us that quantum physics allows to process information dramatically faster than what standard probability theory concedes would gain so much traction.
Furthermore, quantum information does not just hold the promise of faster information processing, that is, roughly speaking, of faster computing, but can also be exploited to devise new and improved means of communication, via faster and more secure communication protocols enabled by the properties of quantum entanglement \highlight{(does entangement specifically help here?)}.

The rationale behind studying this is not just in the fundamental understanding that can be gained from a theoretical point of view. Indeed, ever since Feynman first introduced the idea of \emph{exploiting} the peculiarities of quantum mechanics to solve problems harder to reach within the realm of classical computation~\cite{feynman1982simulating}, more and more research has been devoted into trying to understand exactly why and how quantum mechanics can provide advantages for information processing.
In fact, the realisation itself that this feat was possible at all prompted a stark interest into a broad variety of research directions focused on applying quantum mechanics to build better technologies. This was termed a \textit{second quantum revolution}~\cite{dowling2003quantum}, to differentiate it from the first quantum revolution, which was mostly about understanding how quantum mechanics can be used to explain already observed phenomena.

Among the many approaches explored in the course of the last century, the most notable areas of research fall, broadly speaking, under the names of quantum computation~\cite{shor1997polynomial,steane1998quantum,ladd2010quantum}, quantum simulation~\cite{lloyd1996universal,georgescu2014quantum,koch2019quantum}, quantum communication~\cite{bennett1993teleporting}, and quantum cryptography~\cite{bennett2014quantum}.

Among the other technologies, quantum computers are arguably the idea that received the most traction, possibly due to the monopolising prominence of their classical cousins.
Still, practical and useful quantum computation is still far on the horizon, and will still require a number of theoretical and fundamental breakthroughs to definitively reach~\cite{preskill2018quantum,flamini2018photonic,wang2019integrated}.
Nonetheless, the last few years have seen a number of such breakthroughs follow one another~\cite{fowler2012surface,barends2014superconducting,córcoles2015demonstration,ofek2016extending,arute2019quantum}.
Moreover, it was realised since 2011~\cite{aaronson2011computational} that, even though achieving scalable and fault-tolerant large-scale quantum computation remains a titanic challenge, there are ways to at least settle the overarching question of whether it is possible \textit{even in principle} to solve problems faster than what classical physics allows. This spurred the race to achieve the so-called \textit{quantum computational supremacy}~\cite{preskill2012quantum,gross2013the,aaronson2011computational,bremner2016average,boixo2018characterizing,bouland2018complexity,aaronson2017complexity,neill2018blueprint,arute2019quantum}.

Focusing on quantum computation, one should note the many types of quantum computation schemes and approaches available. From quantum optics, trapped ions, superconducting qubits, cold atoms.
Similarly, many paradigms have been developed. One broad distinction can be made between \textit{discrete variables}~\cite{walmsley2005applied,andersen2015hybrid} and \textit{continuous variables}~\cite{lloyd1999quantum,braunstein2005quantum} approaches.
On top of this, several different paradigms are possible, ranging from the more commonly familiar notion of the circuit model for quantum computation~\cite{nielsen2006quantum}, to one-way quantum computation approaches~\cite{raussendorf2001one,walther2005experimental,browne2006one}, to adiabatic quantum computing~\cite{aharonov2004adiabatic,albash2018adiabatic}.

Integrated photonics~\cite{wang2019integrated,flamini2018photonic}.

\section{Quantum information}

\section{Machine Learning}

\section{Quantum walks}

\subsection{Classical and quantum walks}

An interesting type of computational model that has been the subject of much study is the so-called \textit{quantum walk} model~\cite{aharonov2000quantum,venegas-andraca2012quantum,kempe2003quantum}.

Classical random walks are a type of stochastic processes which have proven to be a powerful technique for the development of stochastic algorithms~\cite{motwani1995randomized,schoning1999probabilistic}, and are used across several different branches of science.
Quantum walks have been introduced in 1993 by Aharonov et al.~\cite{aharonov1993quantum}.
Konno proposed a solid mathematical connection between correlated \acp{QW} and another QW model~\cite{konno2003limit}.

\subsection{DTQWs vs CTQWs}

A first main distinction to make is between \textit{discrete} and \textit{continuous} quantum walks.

DTQWs are systems comprised of a low-dimensional degree of freedom, usually referred to as \textit{coin}, and a high-dimensional one named the \textit{walker}.
These names are in direct analogy with classical random walks. A single \textit{step} in a DTQW consists of a unitary evolution $\calW$ applied to the current state of the QW.
In this model, the time is \textit{discrete} in the sense that there is no real notion of time as a continuous parameter. Rather, the state changes by discrete amounts at every step, just as its classical counterpart proceeds.

A different model is the so-called CTQW. These systems model instead a continuous-time Hamiltonian evolution, governed by a standard Schr\"odinger equation. In a CTQW there is no place for the notion of a \textit{coin} degree of freedom, which makes the structure of these kinds of model rather different, at least on the surface, from their discrete counterpart.

One important application of QWs is their use to develop quantum algorithms.

\subsection{DTQWs}
The simplest type of DTQW is a one-dimensional QW on a line.
In this model, the evolving state lives in a Hilbert space $\calH\equiv\calH_C\otimes\calH_W$, with $\calH_C$ the two-dimensional space of the coin, and $\calH_W$ an high-dimensional space accomodating the possible states of the walker.
A generic state for a DTQW can therefore be written as
\begin{equation}
    \ket\Psi = \sum_j \sum_{s\in\{0,1\}} c_{j,s} \ket{s, j},
\end{equation}
for some coefficients $c_{j,s}\in\CC$, and with the index $j$ spanning the possible values of the walker.

The evolution in a DTQW is a two-step process.
First, a unitary operation is applied locally to the coin. This step is commonly referred to as \textit{coin flipping}, in analogy with the classical case.
Then, a controlled-shift operation is applied to the walker, changing its state conditionally to that of the coin.
Formally we write a single \textit{step operator} as $\calW=\calS\calC$, with $\calS,\calC\in\Lin(\calH)$ and $\calC\equiv\calC\otimes I_W$\footnote{More precisely, this means that $\calC=\calC\otimes I_W$}.
The shift operator $\calS$ is usually written in the form
\begin{equation}
    \calS = \PP_0\otimes \EE_+ + \PP_1\otimes \EE_-,
\end{equation}
where $\PP_k\equiv\ketbra k$ and $\EE_\pm$ are operators moving the walker in one direction or the other, that is,
\begin{equation}
    \EE_+\equiv \sum_k \ketbra{k+1}{k},
    \qquad
    \EE_-\equiv\sum_k\ketbra{k}{k+1}.
\end{equation}


