\chapter{Introduction}

Quantum information science studies how information behaves and can be manipulated at the quantum level.
The reason for studying this is manyfold. One the one hand, quantum mechanics challenged our previously naive way of understanding how information behaves, telling us that the standard rules of probability theory cease to apply in some realms of the real world.
On the other hand, from a more practical point of view, there is much to gain from such a deeper understanding of how information can be manipulated. Indeed, so much of the modern age relying on the capability of manipulating ever vaster heaps of data, it is only natural that an insight telling us that quantum physics allows to process information inconceavably faster than what the standard rules of probability theory concede would gain so much traction.

The rationale behind studying this is not just in the fundamental understanding that can be gained froma theoretical point of view. Indeed, ever since Feynman first introduced the idea of \emph{exploiting} the peculiarities of quantum mechanics to solve problems harder to reach within the realm of classical computation~\cite{feynman1982simulating}, more and more research has been devoted into trying to understand exactly why and how quantum mechanics can provide advantages for information processing.
In fact, the realisation itself that this feat was indeed possible has prompted a stark interest into a broad variety of research directions, for which the term \emph{quantum revolution} was coined~\cite{dowling2003quantum}.

Among the many approaches explored in the course of the last century, the most notable areas of research fall, broadly speaking, under the names of quantum computation~\cite{shor1997polynomial}, quantum simulation~\cite{lloyd1996universal}, quantum communication~\cite{bennett1993teleporting}, and quantum cryptography~\cite{bennett2014quantum}.
\highlight{(second quantum revolution stuff?)}

Focusing on quantum computation.


\section{Quantum information}

\section{Machine Learning}

\section{Quantum walks?}
